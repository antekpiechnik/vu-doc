\section{Conclusions}
\label{sec:conclusion}

In this paper we analyzed, experimented and evaluated the use of \textit{User Activity Data} from \textit{YouTube}
and \textit{Twitter} for generating a user profile.

We have analyzed and identified sets of objects available for extraction from both \textit{structured} and
\textit{unstructured} user activity streams. We were able to select methods of extracting this data and describe
their uses. While \textit{YouTube} provided more organized data, we believe we were able to identify a comparable
amount of UAD for extraction from the \textit{Twitter} service.
We have also compared both user activity sources for generating a user profile. By selecting two methods to focus
on, \ie \textit{named entity recognition} and \textit{extracting relevant user activity}, we were able to concentrate
on obstacles and challenges encountered and propose effective ways of dealing with them (\ie methods for dealing with
\textit{duplicates} and avoiding \textit{false positives}). We have also evaluated our solutions with users, proving
them effective enough for use in a recommender system.

We were able to perform a significant amount of experiments in order to answer both research questions. Due to the time
constraints we were not able to perform experiments on greater amounts of data and using more sophisticated methods.
Despite those restrictions, we have identified multiple ways of extracting UAD from \textit{Twitter} and \textit{YouTube}
as described in section \ref{sec:uad}. We have proved those methods effective enough for generating user profiles, as
described in section \ref{sec:usage_uad}. Basing our conclusions on the limited amount of experiments, we identified
the \textit{YouTube} UAD as a source containing more NEs useful for profile generations. On the other hand, the
user profiles generated from \textit{Twitter} UAD contained more accurate preference information, as evaluated in section
\ref{sec:evaluation}.

We learned that while basic extracting of the user profiles is relatively easy, it is extremely challenging to
guarantee high quality and accuracy of the results. User activity streams from \textit{YouTube} and \textit{Twitter}
required different methods of acquiring and processing user data. These methods have been described in section
\ref{sec:implementation}. We have also noticed that \textit{unstructured} data involves much more complex methods
for generation of the user profile.

We believe that existing and future recommender systems may benefit from our work and focus on improving their results
even further.