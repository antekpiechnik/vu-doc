\section{Conclusions}

In this paper we analyzed, experimented and evaluated the use of \textit{User Activity Data} from \textit{YouTube}
and \textit{Twitter} for generating a user profile.

We have analyzed and identified sets of objects available for extraction from both \textit{structured} and
\textit{unstructured} user activity streams. We were able to select methods of extracting this data and describe
their uses. While \textit{YouTube} provided more organized data, we believe we were able to identify a comparable
amount of UAD for extraction from the \textit{Twitter} service.
We have also compared both user activity sources for generating a user profile. By selecting two methods to focus
on, i.e. \textit{named entity recognition} and \textit{extracting relevant user activity}, we were able to concentrate
on obstacles and challenges encountered and propose effective ways of dealing with them (i.e. methods for dealing with
\textit{duplicates} and avoiding \textit{false positives}). We have also evaluated our solutions with users, proving
them effective enough for use in a recommender system.

We were able to perform a significant amount of experiments in order to answer both research question. Due to the time
constraints we were not able to perform experiments on greater amounts of data and using more sophisticated methods.
Despite that, we believed we were able to answer the posed research questions by providing enough evidence to back our
answers up.

We learned that while basic extracting of the user profiles is relatively easy, it is extremely challenging to
guarantee high quality and accuracy of the results. We have also noticed that \textit{unstructured} data requires
much more complex methods for accurate generation of the user profile but is also able to provide more
accurate data.

We believe that existing and future recommender systems may benefit from our work and focus on improving their results
even further.