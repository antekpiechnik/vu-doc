\section{Discussion of results}
In this chapter we will discuss and compare the results achieved as well as their possible influence on the NoTube project. We will also mention work that might be performed in the future in order to extend this research further.

\subsection{Comparison and analysis of the results}
For both the \textit{YouTube} as well as \textit{Twitter} services, it was extremely challenging to achieve results of
quality. Results varied from quite accurate to containing subjects completely irrelevant to the user being surveyed.

\paragraph{Userbase}
Perhaps the biggest influence on the quality of the results has been the way people use the services researched.
In \textit{YouTube}, users tend to XXXX. On the other hand, Twitter users usually tend to update their statuses
with various information, very rarely regarding TV activities. From the users being analyzed by this poll, a very
little amount of tweets regarded media subjects. Given the fact that those users tweeted on multiple other subjects,
a chance of recognizing an entity in the right context is relatively tiny.

\paragraph{Accuracy of known topics recognition}
Both data sources analyzed returned only a small part of correct topics recognized. In Twitter this score has
averaged around 25\%, whereas in YouTube, XXXX. This means that the data, in small amounts however, is actually present
within those streams. However in order to be able to extract them with greater accuracy, more advanced methods have
to be used.

\paragraph{Accuracy of the preference scores}
Within the properly recognized subjects, methods used by the \textit{Twitter} profiler have proved to be effective enough
to be able to specify whether the user simply knows the subject or could identify preference towards it with quite a high
rate (more than \textit{75\%} preference scores were correct), despite being quite trivial. This also originates in the fact
that Twitter users tend to Tweet about things they like. XXXX

\subsection{Possible influence on the NoTube project}

\subsection{Future work}

\paragraph{Possible improvements of subject recognition methods}
\paragraph{Evolution of the userbase}
\paragraph{Research into other services}

