\section{Potential uses and future work}

In this section, we discuss the future of our work in context of the current
trends among social web services. We also point out potential improvements for
our algorhitms.

\subsection{Possible influence on the NoTube project}

The \textit{NoTube} project is able to personalize recommendations for TV
programmes for a user. In order to maximize it's recommending potential, it
needs to aggregate data from different sources.

Both \textit{YouTube} and \textit{Twitter} certainly do contain information
useful for this topic. However, in order to be able to fully use it's
potential, the recommender would have to incorporate more advanced methods for
media topics recognition and also require an evaluation of information
extracted from those streams based on the current profile of a given user.
Given the little amounts of media-related information in some of user's
streams, results obtained from incorporating such methods might prove not worth
the cost of their implementation.

The \textit{NoTube} target might be users who very rarely employ Twitter or
YouTube for expressing their opinions on media subjects. Moreover, should any
integration with YouTube or Twitter appear within the project, the TV-related
part of a user's stream might become automatically generated, making use of
such services-based recommender slightly obsolete.

On the other hand, there seem to be uses for such recommenders. Provided a user
updates his Twitter information with enough media--related information, a
recommender based on information from this service could be helpful for
creating initial user profiles as well as be able to provide a teaser of the
project's functionality.

\subsection{Evolution of the Userbase}

Since the userbase of social network is constantly changing, we would like to
discuss the possible changes and their influence on this research.

\paragraph{YouTube}

YouTube users tend not to watch many videos related to the TV Shows, we would
like to consider situations where this state might change. An obvious case
appears with Google (the owner of YouTube) planning on offering GoogleTV in the
fall of 2010. Such a move might create opportunities for users to integrate
this service with YouTube, changing the way users employ YouTube.

Furthermore, should YouTube follow the example of services like \textit{Hulu}
and start offering streaming of TV-Shows, it would immediately allow people to
comment, share their preferences, making it a perfect source for extracting
users' preferences.

\paragraph{Twitter}

Recently Twitter has announced plans of bringing advertising (e.g. in a form of
\textit{Promoted Tweets}) to users' stream.  By retweeting such advertisements,
users might show some interest in additional subjects mentioned in their
streams.

What is far more interesting, is the idea of providing Twitter-based
enrichments (that Twitter has also announced), bringing semantic-like concepts
into all tweets and making users able to retweet those concepts and find out
more about them. This could increase the probability of matching entities
correctly.

On the other hand, it seems like Twitter will still mostly be used to update
information on current activities and, in contrast to YouTube, we do not think
this will change soon.

\subsection{Future work}

\subsubsection{Possible improvements of subject recognition methods}

\paragraph{Context/Semantic-based searching for removing accidental matches}

In order to enhance the accuracy of entities recognized within a user's Twitter
stream, one could employ search methods based on analyzing a single tweet's
context regarding other topics related to the media field using semantic
services.  However, this approach could be much slower for multiple tweets
within a significant userbase.

\paragraph{Comparison of TV show entities against more natural language corpora}

In order to minimize the amount of \textit{false positives} matched in a user's
stream, different entities' names and aliases should be ran against corpora
strictly related to the field of research (such as TV or other media related
newspaper/magazine articles). Using this approach while also using semantic
recognition could decrease recommender's speed even further. The eventual
product of employing such approach would have to be measured, as simple methods
of such ''ranking'' proved not effective enough (despite being quite
promising).

\subsubsection{Research into other services}

In order to be able to perform more analaysis, it might be useful to perform
research into other services, such as \textit{Facebook} or \textit{MySpace}.
There however are much more private networks, where users tend not to share
their information publicly. This might make them less available, but due to a
great number of users, they might contain useful data.
