\documentclass{article}
\usepackage[utf8]{inputenc}
\usepackage{multirow}
\usepackage{verbatim}
\usepackage[pdftex]{graphicx}
\usepackage{subfigure}


\begin{document}
\title{\textbf{Profiling users media preferences based on social network data streams}}
\author{Konrad Delong \and Antoni Piechnik}
\date{June 27 2010}

\maketitle

\begin{abstract} Social networks have been becoming hugely popular over the last
years. With enormous number of people sharing information about their lives,
they are now a great source of information about their media interests. Based on both
structured data (such as the social aspect of \textit{YouTube}\footnote[1]{http://www.youtube.com}) as well as unstructured
(\textit{Twitter}\footnote[2]{http://www.twitter.com}), we analyze what data can be extracted from their profiles as well as
how useful such extraction is for profiling those users. Such profiling could
prove hugely influential for media-oriented recommender systems, such as the one
used in the \textit{NoTube}\footnote[3]{http://www.notube.tv} project.
We focus on aggregating and analyzing the publicly available data and covering
different approaches of profile generation
for users. Our experiments reveal multiple ways of employing social networks'
data for users profiling as well as show promising results for possibly employing
this data in a real-life project.
\end{abstract}

\section{Introduction}

Recommender systems used to involve recommending more physical objects \cite{combining-cf-with-pa} to users.
However, more and more abstract topics are becoming parts of such recommenders (such as tags) \cite{accuracy-recommending}. The NoTube project bases its recommendation of TV programmes on aggregating,
extracting and analyzing user activities \cite{notube-main}. The Web provides traces left by users
of social services regarding their activities, varying from user video viewing history
and subscriptions to video channels (e.g. \textit{YouTube})
to users' activity updates formed in natural language (such as \textit{Twitter}) \cite{why-we-twitter}.

In this paper we analyze those data streams and evaluate their usefulness to the \textit{NoTube}
project, by answering two main research questions.

Firstly, we find out \textit{what user data we can collect from both structured and natural language-based social applications?}. We mainly focus on researching the available data and analyze the way people use services such as
\textit{YouTube} and \textit{Twitter} to share their media preferences (Sections 3 and 4).

Furthermore, we look into \textit{how do those services compare when it comes to automated user media
preference profile generating?}. In this part of the paper we present methods of evaluating results of
the aggregation of available data (Section 5) as well as actual user preferences profiling for both
services with a comparative analysis and discussion of the latter (Sections 6 and 7). Our research reveals that despite the
differences between those streams, both of them may provide enough opportunities for user media preference
extraction. We conclude with a summary (Section 8) followed by an overview of the particular implementation methods (Section 9).

\section{Related work}

A considerable number of papers examining social Web services have been
published. Works devoted to YouTube often contain analysis of the video data
stored by the service and its implications to the traffic generated
(\cite{i-tube-you-tube}, \cite{views-from-the-edge},
\cite{statistics-and-social-network}), some study impact of YouTube service on
very narrow topics, like 2006 USA presidential elections
\cite{voters-myspace-youtube}, or social attitude towards vaccinacions
\cite{keelan}. There are also papers analyzing privacy issues of using YouTube
\cite{publicly-private}.

Works analyzing \textit{Twitter} as a data stream can differ from very general \cite{why-we-twitter},
to works devoted to analyzing content of the \textit{Twitter} users' timelines (such as \cite{twitter-content-is-it} and \cite{short-tweet}).

There are also works devoted to automated user profile generation, covering multiple data sources for profile
information aggregation \cite{public-profiles}, as well as abusing social networks for profile generation \cite{twitter-abuse}. Apart from profile generation, a URL recommender system for \textit{Twitter} users
has been introduced by \cite{short-tweet}.

Apart from the papers mentioned, we have also found various implementations of systems that aggregate personal preferences based on different
social services:

\paragraph{Hunch}
\textit{Hunch}\footnote[2]{http://www.twitter.com}) is a service offering recommendations on different topics based on a user's \textit{Twitter} data. It predicts preferences in forms of small questions and is then able to recommend a product or a solution to a problem that is represented with a \textit{Topic} on the site. \textit{Hunch} seems to be more commercial-oriented, e.g. providing users with links to online stores with products they are recommending. However,
it's prediction results suggest that \textit{Twitter} users' timelines contain information about their preferences.

The application can create recommendations based solely on user's \textit{Twitter} data (mostly people they follow). However, when little is available, it requires input from users in order to make future predictions more accurate. It covers topics broader than the \textit{NoTube} project.
\textit{Hunch.com} does not provide information on how the recommender system works, although it provides an API for accessing their data.
Hunch is able to adapt to much more different topics, and it not necessarily focuses on their semantics (rather on what similar people simply like).

\section{Sources of user activities on the Social Web}
For our research we have focused on data from two social services: \textit{YouTube} and \textit{Twitter}.
\subsection{YouTube}
\subsubsection{YouTube data}
YouTube stores various data concerning its users and content. Tables
\ref{ut_video_info} - \ref{ut_channel_info} show the types of information
available.

\begin{table}[ht]
	\begin{tabular}{|p{3cm} | l | p{4cm}|}\hline
		Information & Access & Description\\ \hline

		Title & Public API & \\
		Published & Public API & \\
		Updated & Public API & \\
		Category & Public API & \\
		Tags (keywords) & Public API & \\
		Comments & Public API & \\
		Permissons & Public API & \\
		Description & Public API & \\
		Thumbnails & Public API & Set of video's thumbnails (along with times
		when taken) \\
		Duration & Public API & \\
		Ratings & Public API & Best, worst and average rating, number of votes \\
		Viewcount & Public API & \\
		Favourite count & Public API & \\
		Number of likes & Public API & \\
		Number of dislikes & Public API & \\
		Aspect ratio & Public API & \\
		Related & Public API & \\
		Responses & Public API & \\
		Author & Public API & \\ \hline
	\end{tabular}
	\caption{Information available for a video}
	\label{ut_video_info}
\end{table}

\begin{table}[ht]
	\begin{minipage}[b]{0.5\linewidth}
	\centering
		\begin{tabular}{ | p{3cm} | l |}\hline
		Information & Access \\ \hline
		Number of results & Public API \\
		Search results & Public API \\ \hline
		\end{tabular}
		\caption{Information available for video search results}
	\end{minipage}
	\hspace{0.5cm} % no new lines here!!
	\begin{minipage}[b]{0.5\linewidth}
		\centering
		\begin{tabular}{ | p{3cm} | l |}\hline
			Information & Access\\ \hline
			Uploads & Public API \\
			Gender & Public API \\
			Location & Public API \\
			Age & Public API \\
			Contacts & Public API \\
			Username & Public API \\
			Subscriptions & Public API \\
			Inbox & Public API \\
			Favorites & Public API \\
			History & Screen scraping \\
			Likes & Screen scraping \\
			Issued authentication subtokens & Screen scraping \\ \hline
		\end{tabular}
		\caption{Information available for a user}
	\end{minipage}
\end{table}


\begin{table}[ht]
	\begin{minipage}[b]{0.5\linewidth}\centering
		\begin{tabular}{ | p{3cm} | l |}\hline
			Information & Access \\ \hline
			Created & Public API \\
			Updated & Public API \\
			Author & Public API \\
			Text & Public API \\ \hline
		\end{tabular}
		\caption{Information available for a comment}
	\end{minipage}
	\hspace{0.5cm} % no new lines here!!
	\begin{minipage}[b]{0.5\linewidth}
		\centering
		\begin{tabular}{ | p{3cm} | l |}\hline
			Information & Access \\ \hline
			Demographics & Screen scraping \\
			Referrers & Screen scraping \\
			Countries popularity & Screen scraping \\ \hline
		\end{tabular}
		\caption{Information available for a channel}
		\label{ut_channel_info}
	\end{minipage}
\end{table}


\subsubsection{YouTube usage}

A statistics were performed measuring popularity of three YouTube features: favourites,
subscriptions and uploads. Over a sample of 7500 users most settled down at
relatively low level of activity. All histograms below show numbers of users
(axis y) with $x_1-x_2$ numbers of favourites/subscriptions/uploads. For all
three cases, almost all users belonged to the first histogram range -- the one
with least items. As number of items grew, the number of users decreased so
quickly, that logarithmic scale must have been used in order to make charts
readable.

\begin{figure}[ht]
  \centering
  \subfigure[Favourites]{
		\includegraphics[scale=0.6]{images/favs.png}
		\label{fig:favs}
  }
  \subfigure[Subscriptions]{
		\includegraphics[scale=0.6]{images/subs.png}
		\label{fig:subs}
  }
  \subfigure[Uploads]{
		\includegraphics[scale=0.6]{images/ups.png}
		\label{fig:ups}
  }
  \label{fig:subfigureExample}
  \caption{Histograms of usage of favourites \subref{fig:favs}, subscriptions
  \subref{fig:subs} and uploads \subref{fig:ups}. The x axis represents groups of
  users having $x_1-x_2$ entities, the height of the bars indicates sizes of the
  groups.}
\end{figure}

\newpage
\subsection{Twitter}

Due to the structure of the data available on Twitter, certain approaches had to be undertaken in order to
extract information about their activities related to the media. In this section we would like to describe
available methods to achieve this as well as measurements that may be applied to the aggregated data.
We will use the word \textit{entity} to refer to known media people, shows and programmes.

\subsubsection{Available data}
\paragraph{Mentioning entities names}
Using rather trivial methods of string matching, Twitter data will be analyzed for
mentions of entities' names in tweets. Multiple occurrences of entity names in the
stream indicate interest in the entity mentioned. \\
Entities might be mentioned by:
\begin{itemize}
  \item \textit{full name} - Matching by the entities' full name.
  \item \textit{hashtag} - Matching by a Hashtag form, that is: full name stripped of all whitespace and preceded by a hash (e.g. \textit{\#TheDailyShow})
  \item \textit{twitter username} - Matching by entity's twitter username (if known) preceded by the @ character
  \item \textit{full name acronyms} - Matching acronyms built upon entity names that are longer than three words
\end{itemize}
\paragraph{Usage of preference verbs when tweeting about entities}
For extracting preference information, a vocabulary consisting of
popular preference (both positive and negative) verbs and adjectives has been used.
\paragraph{Usage of activity verbs when tweeting about entities}
Describing the activity of participating in a certain TV experience (such
as watching a show) will also need to be marked as preferential due to the
fact that Twitter users are more likely to specify what they are doing at a
particular moment)
\paragraph{Researching the structure of the most popular mentions}
By retrieving tweets containing the entities, a ranking of most popular words
appearing with the entities will be created and by extracting the most useful
ones, the vocabularies will be updated.
\paragraph{Extracting entities from structured twitter stream sources}
Some applications, such as YouTube of Boxee, can automatically generate tweets
if the user linked their Twitter account with that application. These tweets are
usually very well structured, and therefore very suitable to extract an entity, verb and/or rating from.
\paragraph{Mentioning and following entities}
An attempt will be made for relating the mentions of entities followed (for known Twitter usernames) and the mentions of those and other entities.

\subsubsection{Types of measurements available}
\begin{center}
  \begin{tabular}{ | p{4cm} | p{7cm} | } \hline
    \multicolumn{2}{|c|}{Types of measurements available} \\
    \hline
    \multirow{4}{*} {Mentioning entities}
      & Full name matching \\ \cline{2-2}
      & Matching the twitter username (if known) \\ \cline{2-2}
      & Matching name converted to a hashtag form \\
      & Matching the full name acronym \\ \cline{2-2}
    \hline
    Usage of activity verbs & Mentions using activity verbs \\
    \hline
    \multirow{3}{*}{Using preference verbs}
      & Mentions using preference verbs \\ \cline{2-2}
      & Positive preferences \\ \cline{2-2}
      & Negative preferences \\ \cline{2-2}
    \hline
    \multirow{3}{*}{Following entity}
      & Mentioning the entity by name while following \\ \cline{2-2}
      & Relation between following and actual preferences \\ \cline{2-2}
    \hline
    \multirow{3}{*}{Researching mentions}
      & Statistics of words most used with mentions \\ \cline{2-2}
      & Finding average amount of mentions-to-tweets ratio \\ \cline{2-2}
    \hline
  \end{tabular}
\end{center}

\subsubsection{Corpus used for research}
The example data to be analyzed consists of Twitter streams of over 70 users
consisting in total of about 9000 tweets. They have been selected from the followers of most popular TV channels and broadcasters available on Twitter basing on the amounts of tweets they have accumulated, the amount of their followers and the language they are tweeting in (English in this evaluation).

The most significant reason for using a preselected corpus for this research is
the Twitter API Rate Limiting which makes a wider analysis challenging.
Furthermore, a great deal of Twitter users provide completely irrelevant
information or tweet in languages not useful for this research, which also
influences the effectiveness of a limited Twitter data aggregator. Using a preselected corpus enables measuring and comparing the effectiveness of different counting methods much easier.


\section{Vocabulary}

In order to make the user profile compatible with linked data sources, we used
vocabularies of semantic data from the \textit{Freebase} service, due to it's
well-organized and easy accessible API.

Freebase is a service offering an enormous (over 10 milion topics) amount of semantically linked data on various topics.

The data available on Freebase is generally divided into topics, types and properties. Topics correspond to data freely
available, similar to \textit{Wikipedia}, describing physical entities, artistic creations, abstract concepts etc. In comparison
to \textit{Wikipedia}, all of the data on \textit{Freebase} is semantically structured, allowing users to edit, add and remove data. The \textit{Freebase} project is a part of Linked Open Data cloud\footnote{http://linkeddata.org}, making it a great
entry point for further research.

Types, on the other hand, refer to different aspects of the same topic (\eg \textit{Bob Dylan} is a song writer,
singer and a film actor which all represent different types). All types have properties referring those
types to the topics (such as songs that Bob Dylan has written).

All topics and types have IDs contained within different domains (such as \textit{/en/tina\_fey} for actress \textit{Tina Fey}
or \textit{/medicine/disease} for the type Disease). By referring to those IDs a user is able to search Freebase for
more linked data regarding the interesting topic and type.

From the whole wealth of data, we have focused only on the data related to the media, such as TV Actors, Movies, Feature
actors and TV Programmes as well as Bands and Singers' names.

\section{Extracting user profile from YouTube}

Our goal is to extract a user profile. For this paper, we will assume that user's
profile is a collection of linked data concepts, that should be understood as
user's interests. Many YouTube's videos have very semantic meaning -- they can
be associated with existing concepts (like actors or music bands) by performing
simple text search. In addition to that, analysis of video's tags can give some
more vague overview of user's interests. This kind of data is useful in two
ways: first, it lets easily identify interests for a human, second: it can be
used to verify interests generated with other methods..

In the following sections, we will describe methods used to generate the user
profile. First, we describe our tag counting approach and compare the possible
use of tags and categories in user profiling. Then we will discuss possible
additional information coming from repeating occurrences of the tags in
different sets of videos. The next subsection present the double
nestedness problem, which makes counting tags from subscriptions fundamentally
different that the other two sets. The last subsection describes matching
YouTube videos to actual linked data concepts.

\subsection{Counting YouTube tags and categories}
Analyzing tags information carries some problems. Tags of a single video clip
can be either too numerous, making many
of them irrelevant, too sparse, nonexistent and occasionally just wrong. To avoid
that problem, we aggregate the tags across sets of video clips related to the
profiled user (favourites, subscriptions, uploads). Similar aggregation was
performed for the videos' categories. Below is a comparison of results of counting
tags (chosen by the user) and categories (picked from a small predefined set).


\begin{table}[ht]
\begin{minipage}[b]{0.5\linewidth}\centering

\begin{tabular}{| l | l |}
Category & \# \\ \hline
News & 8 \\
Music & 5 \\
Education & 3 \\
Comedy & 2 \\
\end{tabular}

\end{minipage}
\hspace{0.5cm}
\begin{minipage}[b]{0.5\linewidth}
\centering

\begin{tabular}{| l | l |}
Tag & \# \\ \hline
sonik & 5 \\
bogusław & 5 \\
trzech & 5 \\
kupli & 5 \\
wybory & 4 \\
kraków & 4 \\
polska & 4 \\
\end{tabular}

\end{minipage}

\caption{Most popular tags and categories from account of Bogusław Sonik -- a
polish politician. The word ''wybory'' stands for ''election'' in polish, and
\emph{Trzech kumpli} is a title of a controversial political documentary.}
\end{table}


\begin{table}
\begin{minipage}[b]{0.5\linewidth}\centering

\begin{tabular}{| l | l |}
Category & \# \\ \hline
Music & 53 \\
Comedy & 20 \\
Entertainment & 8 \\
Games & 4 \\
People & 3 \\
Travel & 3 \\
News & 2 \\
\end{tabular}

\end{minipage}
\hspace{0.5cm}
\begin{minipage}[b]{0.5\linewidth}
\centering

\begin{tabular}{| l | l |}
Tag & \# \\ \hline
metal & 17 \\
music & 11 \\
rock & 11 \\
video & 7 \\
the & 7 \\
black & 7 \\
of & 7 \\
\end{tabular}

\end{minipage}

\caption{Most popular tags and categories from anonymous user -- apparently a
heavy metal music fan}
\end{table}

We can see that category count, even though providing general overview of
person's interests (and the ways she uses youtube), cannot be used to find
specific interests. In other words: we can say that a person is more keen to
favourite music videos or comedy videos after looking at his categories, but in
order to say what kind of music the person prefers, we need more specific
information gathered from (e.g.) tags.

\subsection{How to handle duplicates}

When combining tags collected from various sets of videos, a question should be
asked what preference -- if any -- should be given to tags appearing in more
then one of them. Let's analyze some examples. The letters $f$, $s$ and $u$ mean number
of videos in accordingly: favourites, subscriptions and uploads. The
combinations of letters indicate numbers of tags repeating in these sets.

\begin{tabular}{| l | l | l | l | l | l | l | l |}
user & $f$ & $s$ & $u$ & $fs$ & $fu$ & $su$ & $fsu$ \\ \hline
politician & 219 & 32321 & 357 & 116 & 59 & 156 & 34 \\
metal music fan & 1103 & 7570 & 73 & 533 & 27 & 44 & 26 \\
one of authors & 687 & 5553 & 10 & 199 & 2 & 2 & 1 \\
\end{tabular}

The tags that appeared in all three sets are:
\begin{itemize}
  \item{politician: \emph{wybory, europarlament, czerwca, europa, europejska,
  parlament, 18, unia, europejski, kraków, w, 7, pe, 2009, bogusław, po,
  bezpieczeństwo}}
  \item{metal music fan: \emph{head, death, pantera, for, in, nosturi, the, live, cover, metal}}
  \item{one of authors: \emph{film}}
\end{itemize}

As expected, the repeating tags give much more precise view of person's
interests. Unfortunately, the tags get muted when a user is not active in one of
those areas (e.g. does not upload movies very frequently).

\subsection{The double nestedness problem}
Gathering tags across user's favourites or uploads let us treat the whole video
set as a single document, giving each tag equal weight in the counting process.
Taking a similar approach for subscriptions (e.g. giving equal weight to each
tag of each video of every subscription) leads to some unwanted effects.

Suppose a user subscribes to two channels. One represents the british royal
family (133 uploads), the other one: Apple Inc., an american computer company
(19 uploads). The tags referring to new hardware releases would get flooded with
tags regarding british-specific content. In effect, the user's profile would
depend on third parties' youtube activity which is doubtful to be of any
importance to her.

In order to counteract this effect a number of techniques can be used. One of
them is giving each subscription a weight, based on the number of videos or
total number of tags.

\subsection{Identifying entities in youtube streams}
The user profile information would largely benefit from including data aligned
to existing Linked Data sources. An attempt to run this task trivially for
several types of entities (actors, networks) have been performed. The approach
was to identify actors names by a simple text search in videos' titles and
descriptions. Below are sample results (note that descriptions are not shown):

\begin{verbatim}
Rowan Atkinson LIVE: 02 - Fatal Beatings          , Rowan Atkinson
Yes We Can - Barack Obama Music Video             , Scarlett Johansson
Compay Segundo - Chan Chan                        , Compay Segundo
Drivin' Me Wild - Common Ft Lily Allen [OFFICIAL] , Lily Allen
PT 4 - BRITNEY SPEARS 2006 INTERVIEW BEARS ALL    , Britney Spears
16-Lovestoned Live Futuresex/Loveshow             , Justin Timberlake
15- Cry Me A River Live Futuresex/Loveshow        , Justin Timberlake
Jennifer Lopez - Ain't It Funny                   , Jennifer Lopez
Christina Aguilera - Save Me From Myself [Official, Christina Aguilera
Nancy Sinatra Bang Bang                           , Nancy Sinatra
Eddie Vedder - Hard Sun (Official Video)          , Eddie Vedder
\end{verbatim}

The approach is not ideal. However, above certain threshold of repeats, an
assumption that a user likes a given actor becomes more probable.



\section{Using unstructured user data from Twitter}

In this section, we will evaluate the use of entity reference extractions
mentioned in Section 3.2 when ran against the \textit{Twitter} corpus (as described in Section 3.2.1).
We will discuss the results and their usefulness for generating a media preference profile of a user.
We will also cover any problems encountered and suggest possible solutions.

\subsection{Approach}
Since the \textit{Twitter} corpus has been selected as described in the Section 3.2.1,
we decided to cross-validate the measurements against two randomly selected halves of the corpus
to be able to minimize the overfitting of the data. By the measurements, we understand methods
of entity reference extraction described in Section 3.2.The results in the following sections suggest
that entity references are evenly distributed between the two parts of the corpus.

We have used vocabularies consisting of \textit{TV Actors, TV Programmes, Movies and Movie Actors} from
the \textit{Freebase} dataset.

Initial results show that using simple string matching methods we are able to extract interests from
a user's Twitter stream. For an active Twitter user, it holds that up to \textit{18\%} of their tweets
contain mentions of media entities (more on that in section 8.2).

However, there is a great number of entity names (mostly TV Shows) that create noise in the results, \eg show titles such as \textit{Me too}. Removing those false positives has a crucial effect on the eventual accuracy of a Twitter-based profiler
(as we shall see in section 8.3.1).

\subsection{Results}
In this section we present results of locating references to entities using different methods (as described in Section 3.2).
In each paragraph we present results for both halves of the corpus, which we will refer to as \textit{Corpus 1} and
\textit{Corpus 2}.

\subsubsection{Detecting mentioned entities}
\paragraph{Full name matching}
We have measured the average ratio of tweets with entities found to all tweets for different kinds of entities using full name matching of the entity name.

\begin{center}
  \begin{tabular}{ | p{4cm} | p{2cm} | p{1cm}| p{2cm} | } \hline
    Entity (average) & Corpus 1 & & Corpus 2 \\ \hline
    TV Shows & 62.54\% & & 77.21\% \\ \hline
    TV Actors & 0.41\% & & 0.00\% \\ \hline
    Movies & 30.81\% & & 26.19\% \\ \hline
    Movie actors & 0.74\% & & 0.21\% \\ \hline
  \end{tabular} \\
  Table 8: Percentage of all tweets with recognized entities using full name matching \\
\end{center}

Scores presented in Table 8 are used as the a baseline below. We can easily notice high numbers for TV Shows and
Movie entities. As we noticed in our results, most of those matches have been accidental when the entity name is a single
word highly popular in natural language (such as \textit{Love} or \textit{Hotel}. We will refer to those accidental
matches as \textit{false positives} and will discuss avoiding them in section 6.3. Out of half of the corpus, an average of 99 tweets contained a reference to an actor (TV or Movie). This is a low score, but considering the purpose of \textit{Twitter} (as described in section 3.2), we have been expecting them and still believe this amount might be enough for user profiling. On the other hand, the number of such tweets for TV show names is almost 14,500 tweets. This suggests
that \textit{false positives} problem will be a obstacle to achieve accurate results.

\paragraph{Matching twitter usernames}
Due to small number of available Twitter usernames for entities in the \textit{Freebase} dataset (as of writing this paper -- 182 \textit{Twitter} usernames related to TV), this approach is not proving effective. After including those usernames in the search,
we noticed little (0.02\% in on of the corpuses for TV Shows) to almost none increase in the results.

\paragraph{Matching title acronyms}
Only entities with a title of 3 or more words have been used for this measurement, since searching
for smaller acronyms has generated a great amount of noise in the results. We have omitted Actors names, because
they mostly consist of two words and their acronyms rarely correspond to those actors.

\begin{center}
  \begin{tabular}{ | p{4cm} | p{2cm} | p{1cm}| p{2cm} | } \hline
    Entity (average) & Corpus 1 & & Corpus 2 \\ \hline
    TV Shows & 3.07\% & & 2.17\% \\ \hline
    Movies & 2.01\% & & 2.33\% \\ \hline
  \end{tabular} \\
  Table 8: Percentage of all tweets with recognized entities using the entity's name acronym \\
\end{center}

We can easily spot an increase in matched entities, since acronyms such as \textit{SNL}
(for \textit{Saturday Night Live} show) or \textit{BBT} (\textit{Big Bang Theory}) are widely used.
However there might be many misleading acronyms created with this approach. If an acronym is similar
to a natural language word (\eg \textit{CAT}), it will drastically increase the noise and the amount
of \textit{false positives} generated by the matching algorithms (thus the higher percentage of tweets
found). This problem might be solved by methods for avoiding the false positives, which we discuss in section 6.3.

\paragraph{Matching of the name converted to a hashtag form}
For this measurement, all entities' names have been converted to a \textit{hashtag} form (as described
in Section 3.2). In this experiment, also one-word entity names have been used.

\begin{center}
  \begin{tabular}{ | p{4cm} | p{2cm} | p{1cm}| p{2cm} | } \hline
    Entity (average) & Corpus 1 & & Corpus 2 \\ \hline
    TV Shows & 1.18\% & & 2.21 \% \\ \hline
    TV Actors & 0.00\% & & 0.03\% \\ \hline
    Movies & 2.49\% & & 3.12\% \\ \hline
    Movie actors & 0.67\% & & 0.09\% \\ \hline
  \end{tabular} \\
  Table 10: Percentage of all tweets with recognized entities using hashtag formed from the entity name \\
\end{center}

The results in Table 10 let us notice how the person-based entities have either noted a smaller
occurrence rate and the title-based ones have improved. This may be related to the
fact that people usually create hashtags based on a more general entity (such as
a movie) rather then it's specific parts (such as actors that play in it)
when expressing their opinion \cite{edinburg-corpus}

\paragraph{Measuring the frequency of entities in matched tweets}
We have decided to measure how much entities can be found in an average matched tweet. By a matched tweet,
we understand a tweet where a mention of an entity has been found (via any of the methods mentioned above).

\begin{center}
  \begin{tabular}{ | p{4cm} | p{2cm} | } \hline
    Method & Ratio \\ \hline
    Full name & 1.69 \\ \hline
    Acronyms & 1.02 \\ \hline
    Hashtags & 1.13 \\ \hline
  \end{tabular} \\
  Table 11: The average amount of entities found within a tweet \\
\end{center}

The results in Table 11 show that on an average, every time a tweet contains a mention of an entity,
it is likely to contain an additional reference to some other entity. This suggests that users try to add
more references to a tweet when it's already related to media. This might also be helpful for detecting the
context of a single tweet in search of media references.


\subsubsection{Detecting opinion on an entity}
\paragraph{Usage of activity verbs}
For this experiment, a rather small activity verbs vocabulary has been used. (verbs such
as \textit{to watch, to play, to see, to check out, to catch} with their past forms.
\\ Below is a figure showing percentage of tweets using an activity verb
and a entity name (full match) out of all that have been matched with an
entity's name (hence the higher scores).

\begin{center}
  \begin{tabular}{ | p{4cm} | p{2cm} | p{1cm}| p{2cm} | } \hline
    Entity (average) & Corpus 1 & & Corpus 2 \\ \hline
    Movies & 37.4\% & & 24.3\% \\ \hline
    TV Shows & 21.2\% & & 13.4\% \\ \hline
    TV Actors & 1.3\% & & 0.6\% \\ \hline
    Movie actors & 2.1\% & & 0.0\% \\ \hline
  \end{tabular} \\
  Table 12: Average ratio of tweets with activity verbs found to all tweets with entity references. \\
\end{center}

As we can see, the activity verbs are very unlikely to be occurring next to
person-based entities. However, activity verbs are much more popular with both
Movies and TV Shows, which originates from the very idea of Twitter for
updating statuses with information on what the user is currently doing (such as \textit{watching}
a movie).

\paragraph{Usage of preference vocabulary}
Two sets of preference have been used:
\begin{itemize}
  \item positive -- such as \textit{like, recommend, love, great, awesome, stunning, good}
  \item negative -- such as \textit{hate, bad, worse, poor}
\end{itemize}

The Table 13 below shows the general use of preference verbs with occurrences of
mentions.

\begin{center}
  \begin{tabular}{ | p{4cm} | p{2cm} | p{1cm}| p{2cm} | } \hline
    Entity (average) & Corpus 1 & & Corpus 2 \\ \hline
    TV Shows & 7.4\% & & 6.3\% \\ \hline
    Movies & 9.1\% & & 8.4\% \\ \hline
    TV Actors & 0.0\% & & 0.9\% \\ \hline
    Movie actors & 1.2\% & & 2.7\% \\ \hline
  \end{tabular} \\
  Table 13: Average ratio of tweets with preference verbs found to all tweets with entity references \\
\end{center}

We have also measured the positive-to-negative preference ratio (Table 14):

\begin{center}
  \begin{tabular}{ | p{3cm}| p{2cm} | } \hline
    Type & Amount \\ \hline
    Positive & 92\% \\ \hline
    Negative & 8\% \\ \hline
  \end{tabular} \\
  Table 14: The amount of positive and negative verbs found within the tweets with entity references found \\
\end{center}

The amount of preference verbs used whilst mentioning an entity is definitely
smaller compared to activity verbs. However, the positive-to-negative ratio suggests that users'
media preferences expressed on Twitter are mostly positive.

\paragraph{Preference towards entities followed by a user}
By gathering available Twitter usernames from the \textit{Freebase} database,
we were able to perform searches for mentions of those usernames within the followers' streams.

The Table 15 shows the share of different kind of mentions of the specific entity while following.

\begin{center}
  \begin{tabular}{ | p{3cm}| p{2cm} | } \hline
    Match type & Occurence \\ \hline
    Name & 0\% \\ \hline
    Hashtag & 8\% (2 occurrences) \\ \hline
    Username & 92\% (22 occurences) \\ \hline
  \end{tabular} \\
  Table 15: The amount of mentions by different methods of the entity being followed by a user in his stream \\
\end{center}

Since Twitter usernames mostly represent specific people rather than any other kinds of entities, it seems as if users mostly
mention them using their usernames rather than their names in plain or hashtag form.

However, those mentions occur relatively rarely. For a sample of 70 twitter usernames and 5 followers each
(around 4000 tweets), we were able to find out only 24 tweets (about 0.57\%) mentioning directly the people they are following.

Furthermore, following a certain entity should also be regarded just as a preference toward the topics this entity is related to (such as Politics for \textit{BarackObama}). On the other hand, automated locating of more \textit{official} twitter usernames for various entities is challenging, which
limits the use of this approach.

\subsection{Avoiding false positives}
By \textit{false positives}, we understand matches that have been falsely assigned to a user stating user's interest.
They occur usually when a user uses the entity name in a unrelated context when the entity name is similar to a
part of natural language.

In order to reduce the chance of such accidental name matches occurring, entity names were ran against a corpus
of English texts, ranking them by their frequency. The more often an entity name occurs in those corpora, the smaller the chance of it occurring as a name of the TV shows rather than a natural language expression. In order to rate the certainty, a separate function is introduced, taking into account the following:

\begin{itemize}
  \item form of occurrence of the entity name (string match, hashtag)
  \item use of an activity verb
  \item use of a preference verb
  \item the frequency of the entity name in a collection of samples of written and spoken language
\end{itemize}

We expect the first three of those factors to have a huge influence on the accuracy of the entity recognition within a tweet (basing on results in section 6.2). The can see in section 8.3.1 that such approach provided great results for
reducing the amount of false positives.

This measurement is easily translatable into the preference weights to be recorded when profiling a user.
Moreover, by undertaking semantic approaches we are able to analyze different topics found in a tweet and
attempt to rank their relation to the entity we are researching. We will discuss this in the section 9 (Future work).



\section{Results of the extraction}

The following section describes the results of the profile extractions and its
analysis.

\subsection{Analyzed user sets}
In order to analyze the effectiveness of our profilers, we ran them on groups of
users selected from Twitter and YouTube services. The samples were selected
using two different methods. For Twitter, the website's search service was used
in order to identify users that are likely to post updates about media content.
Such an approach let us skip laborious analysis of accounts with hardly any
updates at all, or updates not related to media content in any manner (which,
given Twitter's access limits, was a considerable gain). For YouTube -- service
with no access limits and no user-by-favourites-search function -- a BFS graph
crawl was performed in order to collect the user set.

\subsection{Analysis performed on collected data}

For each user, a profile was generated using our profilers for both Twitter and
YouTube services. We measured effectiveness of profilers as a ratio of number of
items (tweets/videos) with linked data concepts identified to the overall number
of items analyzed. The results are presented in the table below.

\begin{center}
  \begin{tabular}{| l | l | l |}
  Topic & Twitter & YouTube \\ \hline
  Film actors & 0.48\% & 13.12\% \\
  Movie titles & 1.79\% & 45.95\% \\
  TV actors & 0.20\% & 14.73\% \\
  TV shows & 1.5\% & 30.25\% \\
  \end{tabular} \\
  Figure 5: The comparison of the amount of linked data concepts within separate data streams \\
\end{center}

The reader should note, there was no checking of actual correctness of those results at
this stage (the evaluation of results using surveys is presented in the next chapter). We
can still draw some conclusions after this step.

\subsection{Use of text in social services}
YouTube looks to be a more promising area for this kind of search. The
linked data concepts were present in up to 50 times as many items as for the other
web service for every type of concepts searched. This is most probably a
result of different natures of text data in both services.  

Twitter web service doesn't encourage its users to publish their updates in any
specific form. The single requirement is that the length of a single message
was less than 140 characters. For a video description on YouTube: even the name
suggests that this field should be used to characterize the contents
of a video. Such a difference in suggested use might explain the fact, that
much more named entities were found in YouTube stream. Also, a relatively
unrestrained form of Twitter updates is probably the reason for a lower rate of
concepts found in data coming from this web service. Often, instead of properly
mentioning an entity by name Twitter user would decide to use only person's
last name of an acronym of a TV show's title. Behaviours like that result in a
lower effectiveness of text-based entity finder.

\subsection{Variety across concept types}
Another remark worth making is that number of matched items depends on type of
concepts that we are in search of. However, after manual analysis of identified
items, authors came to a conclusion that comparing data stream pervasiveness of
different concepts is counterproductive at this stage. And that is because of
numbers of false positives, which vary wildly depending on the topic. This
observation will be extended in the next chapter.

\subsection{Dependency on amount and quality of data}

The efectiveness of profiler -- not surprisingly -- depends strictly on data
that the profiler was fed. Only a small number of users -- those, whose updates
are published frequently and often feature media-related subjects -- could be
profiled thoroughly. Profiles for similarily small group should be considered
to be only stubs, which would be of relatively little use for recommenders.
Profiles for the rest of users would be essentialy worthless. A technique that
might amortise the effect of small amounts of data would be combining results
coming from various web services. This way, in face of scarce YouTube data, the
profiler might still return meaningful results based on data coming from
Twitter.


\section{Results evaluation}

\subsection{Comparative analysis of YouTube and Twitter as sources of user
profiling data}

\subsubsection{Data availability}
Twitter
- unstructured
- hard to obtain
- limited API

YouTube
- public/unlimited API
- more users

\subsubsection{Type of data extractable}
Twitter
- completely unstructured/mostly irrelevant for the research
- hard obtaining accurate preferences, only interest/activity
- taking slower (text search/indexing)
- hardy applicable NLP methods

YouTube
- mostly language-independent
- structured
- API parts describing preferences (?)

\subsubsection{Usefulness for the NoTube project}

Both services contain large amounts of information useful for recommender
systems, like the one designed for NoTube project. For both of them, the
efficiency for a single user depends mainly on user's activity, and amount of
data generated. YouTube videos offer several ways of identifying their content.
They are also frequently linked from other services. For Twitter, virtually single
way of examining content is through text analysis. On the other hand, Twitter
messages are actually generated by the user who is being profiled. Most of the data
on YouTube comes from other publishers. In other words: after seeing an
interesting television show user is likely to post a Twitter update saying ''I
enjoyed watching X'', while it is rather unlikely that she will search for a
video related to that movie somehow.

YouTube also exposes additional data useful for user profiling: gender, age and
location. In the ``cold start`` problem, those pieces of information are very
often used to narrow down the initial set of recommended items. Twitter is much
more minimalistic in this area -- only email address, full name and the login
name are required for registration. However, it is worth noting, that all
mentioned pieces of information are commonly faked by the users wishing to keep
their privacy.

\subsection{Evaluation of user profiling by users}
\subsubsection{Measuring efficiency of user profiling}

Two ways of measuring efficiency were employed. The first one, was splitting the
input set into training and test data. The algorithm was run on training data,
and results were compared with the testing set. Since testing data set had the
same structure as the training set, this technique was equivalent to running the
same algorithm twice and comparing the results.

This technique was used for analysis of Twitter data. For data imported from
YouTube, this approach was not applicable because of its scarcity. Instead,
manual verification was performed. A survey was performed over a group of 20
people. The survey contained list of topics extracted from user's profile mixed
with random unrelated subjects. Each topic was assigned a mark ranging from 1 to
5 which was supposed to reflect user's interests in particular subject.


\section{Discussion of results}
In this chapter we will discuss and compare the results achieved as well as their possible influence on the NoTube project. We will also mention work that might be performed in the future in order to extend this research further.

\subsection{Comparison and analysis of the results}
For both the \textit{YouTube} as well as \textit{Twitter} services, it was extremely challenging to achieve results of
quality. Results varied from quite accurate to containing subjects completely irrelevant to the user being surveyed.

\paragraph{Userbase}
Perhaps the biggest influence on the quality of the results has been the way people use the services researched.
In \textit{YouTube}, users tend to XXXX. On the other hand, Twitter users usually tend to update their statuses
with various information, very rarely regarding TV activities. From the users being analyzed by this poll, a very
little amount of tweets regarded media subjects. Given the fact that those users tweeted on multiple other subjects,
a chance of recognizing an entity in the right context is relatively tiny.

\paragraph{Accuracy of known topics recognition}
Both data sources analyzed returned only a small part of correct topics recognized. In Twitter this score has
averaged around 25\%, whereas in YouTube, XXXX. This means that the data, in small amounts however, is actually present
within those streams. However in order to be able to extract them with greater accuracy, more advanced methods have
to be used.

\paragraph{Accuracy of the preference scores}
Within the properly recognized subjects, methods used by the \textit{Twitter} profiler have proved to be effective enough
to be able to specify whether the user simply knows the subject or could identify preference towards it with quite a high
rate (more than \textit{75\%} preference scores were correct), despite being quite trivial. This also originates in the fact
that Twitter users tend to Tweet about things they like. XXXX

\subsection{Possible influence on the NoTube project}

\subsection{Future work}

\paragraph{Possible improvements of subject recognition methods}
\paragraph{Evolution of the userbase}
\paragraph{Research into other services}



\section{Implementation details}

\subsection{Technologies used in Twitter research}
\subsubsection{Language}
Most processing has been done using the \textit{Ruby} language. The NoTube tubelets/reasonlets will be implemented in either pure \textit{Java} or \textit{JRuby}.
\subsubsection{Tools}
Twitter API requests are been handled via the OAuth protocol and provided in a JSON-based stream form. Only the newest tweets for the pre-selected Twitter usernames are downloaded. The entities definitions and semantics are provided by the FreeBase database REST interface. All the processing occurs locally on the pre-loaded Twitter data. \\ Ruby gems used include:
\begin{itemize}
  \item \textit{bdb} from mattbauer - a BerkeleyDB wrapper for Ruby
  \item \textit{twitter} from jnunemaker - a wrapper around the Twitter API
  \item \textit{ferret} - full text search engine written in Ruby
  \item \textit{ken} from michael-aufreiter - wrapper for the freebase.org API
\end{itemize}
\subsubsection{Storage}
Due to Twitter API request rate limiting, tweets have been aggregated in a local
\textit{Berkeley DB} database instance providing a fast and easy key-value
information retrieval. Since great amount of data needs to be processed, a full
text search engine has been used, which greatly increased the searching time. In
this case, we decided on using \textit{ferret} - a Ruby implementation of a
full-text search engine (similar to Apache's \textit{Lucene}).

\bibliographystyle{plain}
\bibliography{sources}

\end{document}
