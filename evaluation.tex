\section{User evaluation}
In order to get more detailed measures of our results, we performed user surveys
asking users to judge their extracted profiles. Two groups of users were chosen,
each counting about 15 persons, most of them active users of Twitter and YouTube
web services.

\subsection{The surveys}

Each person was presented a list of topics that were extracted from their online
activity and asked to mark their preference / level on interest in every one of
them. The marks ranged from 0 (dislike), through 1 (neutral) and 2 (like) up to
3 (strongly like). Such scale was suggested for users to be able to quickly respond
and yet provide us with enough data to measure the accuracy of the prediction.
This set was extended with additional mark ''X'', meaning
''don't know this'', which let us quickly identify false positives that sneaked
into extracted profiles.

\subsection{Analyzed data description}
The activeness of users fit in relatively wide range: from 20 up to 400 tweets,
and 6-400 favourites. On average, \textit{2.51\%} of tweets were media-related.
For YouTube, videos useful for profiling formed about one tenth (\textit{9.56\%})
of the analyzed set.

\subsection{Number of false positives}
The number of false positives was astonishingly large. For some users, this
number reached more than 50\% -- maximum of 73.37\% for Twitter using the most accurate method (we discuss
the method accuracy in section 8.3.1) and 83\% for Youtube. This score makes any recommender
algorithm, that is based on such a profile hardly of any use. Luckily, such degenerate
(more than 50\% false positives) cases were relatively uncommon (1 and 2 in accordingly for
Twitter and YouTube surveys) and happened for users with small numbers of items in their streams.

\subsubsection{Removing false positives in Twitter profiling}

We have analyzed the effect of different methods of minimizing the amount of false
positives for Twitter profiling (as described in Section 6.3) by providing test users
polls with results of different methods. The results are presented in the figure below:

\begin{center}
  \begin{tabular}{ | p{4cm} | p{2cm} | p{2cm} | p{2cm} | } \hline
    Method & Entites & False positives & Correct preference \\ \hline
    Full name & 3.11\% & 92.17\% & 42\% \\ \hline
    Full name \& Hashtag & 3.62\% & 80.81\% & 62\% \\ \hline
    Corpus for FN and HT & 1.63\% & 67.25\% & 75\% \\ \hline
    Corpus for FN only & 1.64\% & 61.13\% & 69\% \\ \hline
  \end{tabular} \\
  Figure 6: Results of different matches filtering methods on the results. \textit{FN} stands for Full Name,
  \textit{HT} stands for Hashtag, where \textit{Corpus} means filtering entity names matches by ranking
  their popularity in a natural language corpus. The \textit{Entities} column shows the amount of matched
  entities within all tweets, the \textit{False positives} column shows the percentage of falsely matched
  entities, whereas \textit{Correct preference} shows the percentage of correct preference predictions.
\end{center}

For this experiment we have used the \textit{British National Corpus}\footnote{http://www.natcorp.ox.ac.uk/}.
The table shows that the number false positives without any matching provides us with many matches, less than
10\% being correct. By incorporating Hashtag searching, we have received more results and decreased the amount
false positives (suggesting, as we discussed in Section 6, that topics in form of Hashtags tend to be more
accurate to the interests of the Twitter user who tweets them). After applying corpus filtering, the amount of
entities matched has decreased over twice in size, but we have also noted an decrease in the amount of false positives.
Applying the text corpus ranking only to entities matched by full name allowed us to further decrease the false
positives percentage. This further suggests that Hashtags tend to contain more accurate data regarding
user preferences.

The corpora ranking also has it's downsides. For one of the test users, the corpus ranking method helped to
decrease the false positives by 33\% (from 74\% to 48\%), but also removed two \textit{accurate} matches (marked as
accurate in the non-filtered polls -- a Movie \textit{The Weekend} and a TV Show called \textit{I'm telling}).
However, since the accuracy is the most important aspect of user profiling, employing this method is worth
the risk.

\subsection{Preference rate accuracy}
After counting out false positives, the correctness of ratings was judged. The
number of rates guessed correctly by the Twitter profiler revolving around \textit{70\%},
while only \textit{50\%} of YouTube rate guesses turned out to be correct. This
might suggest that the frequency of tweeting on a given subject as well as preference vocabulary used
is a better indicator of interest than a number of videos in user's favourites feed.
