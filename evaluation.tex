\section{Results evaluation}

\subsection{Comparative analysis of YouTube and Twitter as sources of user
profiling data}

\subsubsection{Data availability}
Twitter
- unstructured
- hard to obtain
- limited API

YouTube
- public/unlimited API
- more users

\subsubsection{Type of data extractable}
Twitter
- completely unstructured/mostly irrelevant for the research
- hard obtaining accurate preferences, only interest/activity
- taking slower (text search/indexing)
- hardy applicable NLP methods

YouTube
- mostly language-independent
- structured
- API parts describing preferences (?)

\subsubsection{Usefulness for the NoTube project}
Twitter
- depending on user's activity, might actually contain more information (describing media activity in TV and Theatres)
- profiles way less accurate than YouTube

YouTube
- ?? (more short-movies, music and stuff)

\subsection{Evaluation of user profiling by users}
\subsubsection{Measuring efficiency of user profiling}
XXX
