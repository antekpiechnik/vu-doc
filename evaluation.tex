\section{Results evaluation}

\subsection{Comparative analysis of YouTube and Twitter as sources of user
profiling data}

\subsubsection{Data availability}
The greatest issue with generating user profiles is the availability of the data. This covers not only access to remotely
stored data, but also how information on user preferences is stored within it.
\\ The Twitter data source is challenging to process due to very restrictive API Access Limit Rates, making closed applications
unable to work at full speed. On the other hand, YouTube has a mostly complete (except for e.g. user's history) public API,
access to which is virtually unrestricted and unlimited for researching systems. This is a great advantage, allowing to use
much more approaches and aggregate more data for a greater deal of users.
\\ Yet the greatest difference between those two sources is the internal structure (or lack thereof) of atomic information
available. YouTube has all of it's data perfectly organized (such as tags for videos, tags for channels, channels users are
subscribed to, etc.) and revolving around videos, whereas on Twitter, a single user can post any kind of text that is
less or equal than 140 characters in length, making extraction of specific preference data nontrivial.

\subsubsection{Type of data extractable}
Due to great diversity of data stored in a Twitter users' streams, only a little it's part can be effectively used for profiling
users. That forces any kind of research to start with locating useful data. This however is not much different from YouTube, where
a lot of users have data almost completely inadequate for any kind of research.
\\ Furthermore, the language of streams is also a factor, since it is difficult to extract preferences from Non-English Twitter
streams. The same problem applies to YouTube, despite it's API being already structured. A great deal of tags is English,
but this may also vary depending on both user's and video's language, hence it is useful to detect their language before
proceeding any further. Moreover in Twitter, analyzing an update's usefulness, any script will need to apply text searching (mostly
unnecessary in YouTube), which also makes the whole process slower.
\\ YouTube's data is, as already mentioned, well organized. That means, as soon as a useful user is found, it is not hard to
extract media preference information (such as watched/liked videos' tags) stored for them on the server. On the other hand,
any kind of attempt to extract preference data from Twitter carries the risk of being inaccurate, due to it's complete lack of structure.
Since all tweets are limited to 140characters, they are usually formed as incorrect sentences, forcing to abandon traditional
NLP methods.

\subsubsection{Usefulness for the NoTube project}

\subsection{Evaluation of user profiling by users}
\subsubsection{Measuring efficiency of user profiling}
XXX
