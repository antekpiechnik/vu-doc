\section{Introduction}

Recommender systems used to involve recommending more physical objects \cite{combining-cf-with-pa} to users.
However, more and more abstract topics are becoming parts of such recommenders (such as tags) \cite{accuracy-recommending}.
The NoTube project bases its recommendation of TV programs on aggregated, extracted and analyzed user activities
\cite{notube-main}.

On the Web, a great deal of \textit{user activity data} (UAD) can be found on several networks. This UAD varies from user
video viewing history and video channels subscriptions, \eg on \textit{YouTube}, to users' activity updates formed
in natural language, such as \textit{Twitter} \cite{why-we-twitter}.

In this paper we experiment on those data streams and evaluate their usefulness to the user profile generation. We answer
two main research questions:
\begin{enumerate}
  \item What user data can we collect from \textit{structured} and \textit{unstructured} user activity streams?
  \item How useful is \textit{structured} UAD as compared to \textit{unstructured} UAD, to generate a user profile?
\end{enumerate}

By the \textit{structured} UAD, we understand data organized in predefined
preference groups, such as favorites, subscriptions, etc. on \textit{YouTube}.
We define \textit{unstructured} UAD as data expressed in natural language, \eg user streams on \textit{Twitter}.

To answer the first research question, we focus on analyzing the
data and the way people use services like \textit{YouTube} and \textit{Twitter} to share
their media preferences.

We answer the second research question with a comparative analysis of the available data and
methods for generating user profiles for both services.
Our research reveals that more complex methods are required when processing \textit{unstructured} UAD.
We have also observed that there are differences in user profiling data as well as its
accuracy. However, despite those differences, both of them may provide
enough opportunities for user preference extraction.

In the following section we describe related work on the subject. In section \ref{sec:vocabulary}, we describe the
vocabulary we employed in our experiments. Section \ref{sec:uad} contains a description of a selection of sources
of user activities on the Social Web. We follow by detailing the results of our experiments
with the usage of UAD, describing methods applied, challenges encountered and generating a user profile in
section \ref{sec:usage_uad}. Section \ref{sec:evaluation} details the user evaluation of generated profiles.
We conclude with a section on future developments in section \ref{sec:future}, our implementation details
in section \ref{sec:implementation} and a summary of our work in section \ref{sec:conclusion}

\section{Related work}

\paragraph{The problem of generating semantic user profile} is widely covered in the
research area. Maria Golemati et al. created the basic ontology for describing
user profile data \cite{creating-ontology-for-user-profile}. It proved to be
very suitable for static data concerning users. However -- not particularly
so for data that changes frequently. Our profilers describe elements of user
profile, that change on a regular basis -- their interests.  

Fabian Abel et al. showed in their
paper ''Interweaving Public User Profiles on the Web'' \cite{public-profiles},
that systems that gather data from \textit{several} social web services are
able to draw further-reaching conclusions, than systems that analyze the web
services content separately for a single service. They crawled 160,000 user
accounts using Google Social Graph API (more on that below) and presented
profiles with data combined from several user accounts.  The extractors, that
were prepared during our research will find use in creating such a system.

\paragraph{Twitter and YouTube data analysis} Many papers on \textit{Twitter}
and \textit{YouTube} contain sets of statistics serving the goal of
deeper understanding of the nature of those services and patterns in which
\textit{Twitter} and \textit{YouTube} are used. A. Java et al. \cite{why-we-twitter}
describe, what they call ''community intentions'', finding out, that main reasons for
Twitter use are reporting their daily activities and seeking/sharing information. This
is beneficial information for our work, since both of these purposes may involve media
related content, which we focus on. Mor Naaman et al. \cite{twitter-content-is-it}
extended this work, by performing human computation and dividing the categories even further.
The two of three most popular kinds of \textit{Twitter} updates they identified were
''Me Now'' -- informing about current activities, and ''Opinions/Complains''.
Again, both potentially contain media-related content.

Among works analyzing YouTube, most contain statistics concerning the video
data gathered on the site, and the implications of the service on the network
traffic \cite{i-tube-you-tube}, \cite{views-from-the-edge},
\cite{statistics-and-social-network}. One of them
\cite{statistics-and-social-network} contains -- among other experiments --
analysis of video length distribution for most popular categories.  While the
curve shapes look similar for most of them, there is a noticeable difference
among videos categorized as ''Music''. The most popular time-span of the videos
there is 3 to 4 minutes -- a typical length of a music clip. This suggests,
that many of the clips in this category might be videos that contain a single song.

\paragraph{Web-Based Graphs} There are several attempts at sneaking additional
computer-processable semantic data into regular human-readable web pages.  The
simplest of these are \textit{microformats}. The idea is to use well-defined
markup names, so that others could analyze, and possibly provide additional
features based on that. Several microformats for social activity, and semantic
links between the pages, are heavily promoted. The information stored in
websites, that contain these micro-formatted pieces of data, forms a semantic
graph, which we will call a Web-Based Graph (WBG).  
The \textit{Google Social Graph} is an attempt at grasping
the semantic data published across many social web sites. Namely: it uses a web
crawler to find and recognize FOAF (ontology) and XFN (a \textit{microformat}) profiles,
and gives API access to anyone who is willing to explore that resources.

Another WBG that gains popularity in the recent days, is Facebook's Open Graph
Protocol. It lets webmasters integrate their sites with Facebook, by giving them
identity as nodes in a global graph, that the company uses for their
web service. As the graph data is hidden within web pages' \textit{meta tags},
it is freely available for everyone to access.

One could expect, that one of the goals of introducing these technologies will
be to prepare user profiles based on semantic knowledge accumulated from across
the web.  This resembles topic of our research, and we expect a number of works
on profiling based of WBGs to appear in the future, as the
popularity of such technologies arises.

\paragraph{Semantic recommender systems} Our work will be used to generate input
for a media-related semantic recommender system. Several systems like this have
been published, covering, however, other thematic areas, like works of art and
antics \cite{museums} or web search \cite{social-tagging}. An integral part of
these solutions is preparation of a user model. The user model employed in
\cite{museums} uses a similar scale for describing person's relation to an item.
The degree of likeness ranges from ''I hate it'' to ''I like it very much''
across 5 levels. The scale we decided to use limits the negative marks to
a single level only, and provides another mark that explicitly specifies user's unawareness
of and item.

Also worth mentioning here is the proprietary solution
\textit{Hunch}\footnote{http://www.hunch.com}. It is a web service that offers
recommendations on different topics based on a user's \textit{Twitter} data. As
opposed to the WBG solutions, it limits itself to a single source. Yet, it is
able to to predict its users' answers to various questions based solely on their
\textit{Twitter} updates, followers and following lists.
In addition to that, it is able to recommend products
to users and provide them with links to online stores containing those products.
Initial prediction results and their accuracy suggest that
\textit{Twitter} user data does contain information about their preferences.
\textit{Hunch} is a \textit{gray box}
--  it does not provide information on how the recommender system works, but it
provides a rich API for accessing its data.
