\section{Introduction}

The NoTube project bases its recommendation on aggregating, extracting and analyzing user activities. Since media oriented web services (such as YouTube) hold a great amount of information regarding a users' video viewing preferences, it might provide a great deal of additional information for profiling NoTube users. General activity services (social networks like Facebook or Twitter) detail the lives of its users, it is very likely that they also contain information regarding their tv watching activities.

The most important question is what user data we can collect from both structured and unstructured social applications? When it comes to YouTube, users can mark videos as favorite, subscribe to other users' channels and comment on videos. From all that information we can not only extract the titles of videos they are interested in but also tags describing them as well as relations to other videos. On the other hand, unstructured data streams like Twitter contain much more irrelevant information, but can enable us to extract a greater amount of activities limited only by what the natural language offers. Apart from extracting the activities, we will be able to relate the entities (like TV programme and channel names). A user profile could also be partially generated from a user's activity frequency as well as the users they are following and the hash tags used.

Another part of the project will cover finding ways of measuring the accuracy and effectiveness of profiling a user based on those social data streams. We would like to find out how efficient browsing through different networks is when it comes to generating those profiles, both in terms of the quality of information regarding the media preferences as well as its amount relative to the amount of data collected. We will attempt to find a perfect ratio between the level of detail of the information we extract and the quality of user profiles generated. We will evaluate interest counting strategies for different types of entities recognized in various contexts.

\subsection{Related work}
Patricia G. Lange in her article ''Publicly Private and Privately Public: Social
Networking on YouTube'' discusses the social behaviours of YouTube users and
consequences of these behaviours to users' privace.

Xu Cheng, Cameron Dale and Jiangchuan Liu in ''Statistics and Social Network of
YouTube Videos'' analyze features of videos from a large set gathered through
several automated crawling sessions.

Jennifer Keelan, Vera Pavri-Garcia, George Tomlinson and Kumanan Wilson
performed an analysis of YouTube content with focus on the topic of
immunization, in order to prepare physicians to respond to conspiracy theories
published there.

The article "Voters, MySpace, and YouTube", by Vassia Gueorguieva, discusses use
of internet media platforms (YouTube among these) in 2006 presidential elections
in USA. It contains analysis of their usefulness in predicting the final
results, and their impact on the results themsevles.

Meeyoung Cha, Haewoon Kwak, Pablo Rodriguez, Yong-Yeol Ahn, and Sue Moon in
their paper "I Tube, You Tube, Everybody Tubes: Analyzing the World’s Largest
User Generated Content Video System" perform in-depth analysis of YouTube views
patterns, mostly from the point of view of ISPs and content providers.


