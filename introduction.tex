\section{Introduction}

Recommender systems used to involve recommending more physical objects \cite{combining-cf-with-pa} to users.
However, more and more abstract topics are becoming parts of such recommenders (such as tags) \cite{accuracy-recommending}. The NoTube project bases its recommendation of TV programmes on aggregating,
extracting and analyzing user activities \cite{notube-main}.

On the Web, a great deal of \textit{user activity data} can be found on several networks. This UAD (\textit{user activity data}) varies from user video viewing history and subscriptions to video channels (\eg \textit{YouTube}) to users' activity updates formed in natural language (such as \textit{Twitter}) \cite{why-we-twitter}.

In this paper we experiment on those data streams and evaluate their usefulness to the user profile generation. We answer
two main research questions:
\begin{enumerate}
  \item What user data can we collect from \textit{structured} and \textit{unstructured} user activity streams?
  \item How useful is \textit{structured} UAD as compared to \textit{unstructured} UAD, to generate a user profile?
\end{enumerate}

By the \textit{structured} user activity data, we understand data organized in predefined
preference groups (such as favorites, subscriptions, etc. on \textit{YouTube}).
We define \textit{unstructured} UAD, on the other hand, as data expressed in natural language
(user streams on \textit{Twitter}).

To answer the first research question, we focus on analyzing the
data and the way people use services like \textit{YouTube} and \textit{Twitter} to share
their media preferences.

We answer the second research question with a comparative analysis of the available data and
methods for generating user profile for both services.
Our research reveals that more complex methods are required when processing \textit{unstructured} UAD.
We have also observed that there are differences in user profiling data as well as its
accuracy. However, despite those differences, both of them may provide
enough opportunities for user media preference extraction.

In the following section we describe related work on the subject. In section 3, we describe
a selection of sources of user activities on the Social Web. Section 4 contains the description of
the vocabulary we employed in our experiments. We follow by detailing the results of our experiments
with the usage of \textit{structured} (section 5) and \textit{unstructured} (section 6) user activity data.
The next part of the paper (section 7) is a comparison of the usage of both kinds of UAD. Section 8
details the user evaluation of generated profiles. We conclude with a section on future
work, a summary (section 9) and our implementation details (section 10).

\section{Related work}

1. O profilowaniu: problem-method-results
2. O twitterze/U2bie
3. Grafy sieciowe

\paragraph{The problem of generating semantic user profile} is widely covered in the
research area. Maria Golemati et al. created the basic ontology for describing
user profile data \cite{creating-ontology-for-user-profile}. It proved to be
very suitable for static data concerning users, however -- not particularily
helpful for data that changes frequently. Fabian Abel et al. showed in their
paper ''Interweaving Public User Profiles on the Web'' \cite{public-profiles},
that systems that gather data from \textit{several} social web services are
able to draw further-reaching conclusions, than systems that analyze the web
services content separately for each service. They crawled 160,000 user
accounts using Google Social Graph API (more ob that below) and presented
profiles with data combined from several user accounts.  The extractors, that
were prepared during our research might find use in creating such a system.

\paragraph{Twitter and YouTube}. There have been several papers published
mentioning directly the services we analyzed. Many of them contain sets of
statistics serving the goal of deeper understanding their nature and patterns
in which they are used. A. Java et al. \cite{why-we-twitter} describe, what
they call ''community intensions'', finding out, that main reasons for Twitter
use are reporting their daily activities and seeking/sharing information. Both
of these activities can pass media related content (which is what this paper is
focused on). Mor Naaman et al. \cite{twitter-content-is-it} extended their
work, by performing human computation and dividing the categories even further.
The two of three most popular kinds of twitter updates they identified were
''Me Now'' -- informing about current activities, and ''Opinions/Complains''.
Again, both -- depending on the user, though -- are very likely to contain
media-related content.

Among works analyzing YouTube, most contain statistics concerning the video
data gathered on the site, and the implications of the service on the network
traffic \cite{i-tube-you-tube}, \cite{views-from-the-edge},
\cite{statistics-and-social-network}. One of them
\cite{statistics-and-social-network} contains -- among other experiments --
analysis of video length distribution for most popular categories.  While the
curve shapes look similar for most of them, there is a noticeable difference
among videos categorized as ''Music''. The most popular time-span if the videos
there is 3 to 4 minutes -- a typical length of a music clip. This suggests,
that many of the clips in this category might be considered as a musical track
representation.

\paragraph{Web-based graphs}. There are several attempts at sneaking additional
computer-processable semantic data into regular human-readable web pages.  The
simplest of these are \textit{microformats}. The idea is to use well-defined
markup names, so that others could analyze, and possibly provide additional
features based on that.  The \textit{Google Social Graph} is an attempt at grasping
the semantic data published across many social web sites. Namely: it uses a web
crawler to find and recognize FOAF (ontology) and XFN (a microformat) profiles,
and gives API access to anyone who is willing to explore that resources. A
similar initiative, more vendor specific, however is Facebook's Open Graph
Protocol. It lets webmasters integrate their sites with Facebook, giving them
identity as nodes in a global graph, that the company uses for their
web service. As the graph data in hidden within web pages \textit{meta tags},
it is freely available for everyone to access.

\paragraph{Hunch}
\textit{Hunch}\footnote{http://www.hunch.com} is a service offering recommendations on different topics based on a user's
\textit{Twitter} data. It is able to employ methods of graph reasoning for users followers and following lists
in order to predict his answers to various questions. In addition to that, it is able to recommend products
to users and provide them with links to online stores containing those products.
Initial prediction results and their accuracy suggest that
\textit{Twitter} users data does contain information about their preferences. \textit{Hunch} is a \textit{gray box}
--  it does not provide information on how the recommender system works, but it provides a rich
API for accessing its data
