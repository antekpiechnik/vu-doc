\section{Introduction}

Recommender systems used to involve recommending more physical objects \cite{combining-cf-with-pa} to users.
However, more and more abstract topics are becoming parts of such recommenders (such as tags) \cite{accuracy-recommending}. The NoTube project bases its recommendation of TV programmes on aggregating,
extracting and analyzing user activities \cite{notube-main}.

On the Web, a great deal of \textit{user activity data} can be found on several networks. This UAD (\textit{user activity data}) varies from user video viewing history and subscriptions to video channels (\eg \textit{YouTube}) to users' activity updates formed in natural language (such as \textit{Twitter}) \cite{why-we-twitter}.

In this paper we experiment on those data streams and evaluate their usefulness to the user profile generation. We answer
two main research questions:
\begin{enumerate}
  \item What user data can we collect from \textit{structured} and \textit{unstructured} user activity streams?
  \item How useful is \textit{structured} UAD as compared to \textit{unstructured} UAD, to generate a user profile?
\end{enumerate}

By the \textit{structured} user activity data, we understand data organized in predefined
preference groups (such as favorites, subscriptions, etc. on \textit{YouTube}).
We define \textit{unstructured} UAD, on the other hand, as data formed in freely in
natural language (user streams on \textit{Twitter}).

To answer the first research question, we mainly focus on analyzing the available
data and the way people use services such as \textit{YouTube} and \textit{Twitter} to share
their media preferences.

We answer the second research question by presenting methods for evaluating results of
available data aggregation and user profiling for both services with a comparative analysis.
Our research reveals that more complex methods are required when processing \textit{unstructured} UAD.
We have also observed that there are differences in the amount of user profiling data as well as its
accuracy. However, despite those differences, both of them may provide
enough opportunities for user media preference extraction.

In the following section we describe related work on the subject. In section 3, we describe
a selection of sources of user activities on the Social Web. Section 4 contains the description of
the vocabulary we employed in our experiments. We follow by detailing the results of our experiments
with the usage of \textit{structured} (section 5) and \textit{unstructured} (section 6) user activity data.
The next part of the paper (section 7) is a comparison of the usage of both kinds of UAD. Section 8
details the user evaluation of generated profiles. We conclude with a section on future
work, a summary (section 9) and our implementation details (section 10).

\section{Related work}

A considerable number of papers examining social Web services have been
published. Works devoted to YouTube often contain analysis of the video data
stored by the service and its implications to the traffic generated
(\cite{i-tube-you-tube}, \cite{views-from-the-edge},
\cite{statistics-and-social-network}), some study impact of YouTube service on
very narrow topics, like 2006 USA presidential elections
\cite{voters-myspace-youtube}, or social attitude towards vaccinacions
\cite{keelan}. There are also papers analyzing privacy issues of using YouTube
\cite{publicly-private}.

Works analyzing \textit{Twitter} as a data stream can differ from very general \cite{why-we-twitter},
to works devoted to analyzing content of the \textit{Twitter} users' timelines (such as \cite{twitter-content-is-it} and \cite{short-tweet}).

There are also works devoted to automated user profile generation, covering multiple data sources for profile
information aggregation \cite{public-profiles}, as well as abusing social networks for profile generation \cite{twitter-abuse}. Apart from profile generation, a URL recommender system for \textit{Twitter} users
has been introduced by \cite{short-tweet}.

Apart from the papers mentioned, we have also found various implementations of systems that aggregate personal preferences based on different
social services:

\paragraph{Hunch}
\textit{Hunch}\footnote{http://www.hunch.com} is a service offering recommendations on different topics based on a user's \textit{Twitter} data. It predicts preferences in forms of small questions and is then able to recommend a product or a solution to a problem that is represented with a \textit{Topic} on the site. \textit{Hunch} seems to be more commercial-oriented, \eg providing users with links to online stores with products they are recommending. However,
it's prediction results suggest that \textit{Twitter} users' timelines contain information about their preferences.

The application can create recommendations based solely on user's \textit{Twitter} data (mostly people they follow). However, when little data is available, it requires input from users in order to make future predictions more accurate. It covers topics broader than the \textit{NoTube} project.
\textit{Hunch.com} does not provide information on how the recommender system works, although it provides an API for accessing their data.
Hunch is able to adapt to much more different topics, and it does not necessarily focus on their semantics (rather on what similar people simply like).