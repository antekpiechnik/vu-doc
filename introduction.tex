\section{Introduction}

Recommender systems used to involve recommending more physical objects \cite{combining-cf-with-pa} to users.
However, more and more abstract topics are becoming parts of such recommenders (such as tags) \cite{accuracy-recommending}. The NoTube project bases its recommendation of TV programmes on aggregating,
extracting and analyzing user activities \cite{notube-main}. The Web provides traces left by users
of social services regarding their activities, varying from user video viewing history
and subscriptions to video channels (e.g. \textit{YouTube})
to users' activity updates formed in natural language (such as \textit{Twitter}) \cite{why-we-twitter}.

In this paper we analyze those data streams and evaluate their usefulness to the \textit{NoTube}
project, by answering two main research questions.

Firstly, we find out \textit{what user data we can collect from both structured and natural language-based social applications?}. We mainly focus on researching the available data and analyze the way people use services such as
\textit{YouTube} and \textit{Twitter} to share their media preferences (Section 3).

Furthermore, we look into \textit{how do those services compare when it comes to automated user media
preference profile generating?}. In this part of the paper we present methods of evaluating results of
the aggregation of available data (Sections 5 and 6) as well as actual user preferences profiling for both
services with a comparative analysis and discussion of the latter (Sections 7 and 8). Our research reveals that despite the
differences between those streams, both of them may provide enough opportunities for user media preference
extraction. We conclude with a summary and future work (Section 8).

\section{Related work}

A considerable number of papers examining social Web services have been
published. Works devoted to YouTube often contain analysis of the video data
stored by the service and its implications to the traffic generated
(\cite{i-tube-you-tube}, \cite{views-from-the-edge},
\cite{statistics-and-social-network}), some study impact of YouTube service on
very narrow topics, like 2006 USA presidential elections
\cite{voters-myspace-youtube}, or social attitude towards vaccinacions
\cite{keelan}. There are also papers analyzing privacy issues of using YouTube
\cite{publicly-private}.

Works analyzing \textit{Twitter} as a data stream can differ from very general \cite{why-we-twitter},
to works devoted to analyzing content of the \textit{Twitter} users' timelines (such as \cite{twitter-content-is-it} and \cite{short-tweet}).

There are also works devoted to automated user profile generation, covering multiple data sources for profile
information aggregation \cite{public-profiles}, as well as abusing social networks for profile generation \cite{twitter-abuse}. Apart from profile generation, a URL recommender system for \textit{Twitter} users
has been introduced by \cite{short-tweet}.

Apart from the papers mentioned, we have also found various implementations of systems that aggregate personal preferences based on different
social services:

\paragraph{Hunch}
\textit{Hunch}\footnote[2]{http://www.hunch.com}) is a service offering recommendations on different topics based on a user's \textit{Twitter} data. It predicts preferences in forms of small questions and is then able to recommend a product or a solution to a problem that is represented with a \textit{Topic} on the site. \textit{Hunch} seems to be more commercial-oriented, e.g. providing users with links to online stores with products they are recommending. However,
it's prediction results suggest that \textit{Twitter} users' timelines contain information about their preferences.

The application can create recommendations based solely on user's \textit{Twitter} data (mostly people they follow). However, when little is available, it requires input from users in order to make future predictions more accurate. It covers topics broader than the \textit{NoTube} project.
\textit{Hunch.com} does not provide information on how the recommender system works, although it provides an API for accessing their data.
Hunch is able to adapt to much more different topics, and it not necessarily focuses on their semantics (rather on what similar people simply like).