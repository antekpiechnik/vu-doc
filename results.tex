\section{Results of the extraction}

The following section describes the results of the profile extractions and its
analysis.

\subsection{Analysed user sets}
In order to analyze the effectiveness of our profilers, we ran them on groups of
users selected from Twitter and YouTube services. The samples were selected
using two different methods. For Twitter, the website's search service was used
in order to identify users that are likely to post updates about media content.
Such an aproach let us skip laborous analysis of accounts with hardly any
updates at all, or updates not related to media content in any manner (which,
given Twitter's access limits, was a considerable gain). For YouTube -- service
with no access limits and no user-by-favourites-search function -- a BFS graph
crawl was performed in order to collect the user set.

\subsection{Analysis performed on collected data}

For each user, a profile was generated using our profilers for both Twitter and
YouTube services. We measured effectiveness of profilers as a ratio of number of
items (tweets/videos) with linked data concepts identified to the overall number
of items alanyzed. The results are presented in the table below.


\begin{tabular}{| l | l | l |}
Topic & Twitter & YouTube \\ \hline
Film actors & 4.70\% & 10.92\% \\
TV actors & 4.34\% & 9.12\% \\
TV shows & 2.3\% & 4.38\% \\
\end{tabular}


The reader should note, there was no checking of actual correctness of those results at
this stage (the evaluation of results using surveys is in the next chapter). We
can still draw some conclusions after this step.

\subsection{Use of text in social services}
First: YouTube looks to be a more promising area for this kind of search. The
linked data concepts were present in about twice as many items as for the other
web service for almost every type of concepts searched. This is most probably a
result of different natures of text data in both services.  Text in a Twitter
update is an expressive or informative message The YouTube video's description's
raison d'être is to characterize contents of the clip.

\subsection{Variety across concept types}
Another remark worth making is that number of matched items depends on type of
concepts that we are in search of. However, after manual analysis of identified
items, authors came to a conclusion that comparing data stream penetration of
different concepts is counterproductive at this stage. And that is because of
numbers of false positives, which vary wildly depending on the topic. This
observation will be extended in the next chapter.

\subsection{Dependency on amount of data}
The efectiveness of profiler -- not surprisingly -- depends strictly on data
that the profiler was fed. For vast majority of users (see section 2.2) the
profiles will be fed hardly any data at all. As the reader just learned, only a
small fraction of this data is useful for profilers. That leads us to a
conclusion that only a small number of very active users could be profiled
thoroughly. Profiles for similarily small group should be considered to be only
stubs, which would be of relatively small use for recommenders. Profiles for the
rest of users would be essentialy worthless.
