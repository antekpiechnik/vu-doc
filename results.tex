\section{Results of the extraction}

The following section describes the results of the profile extractions and its
analysis.

\subsection{Analyzed user sets}
In order to analyze the effectiveness of our profilers, we ran them on groups of
users selected from Twitter and YouTube services. The samples were selected
using two different methods. For Twitter, the website's search service was used
in order to identify users that are likely to post updates about media content.
Such an approach let us skip laborious analysis of accounts with hardly any
updates at all, or updates not related to media content in any manner (which,
given Twitter's access limits, was a considerable gain). For YouTube -- service
with no access limits and no user-by-favourites-search function -- a BFS graph
crawl was performed in order to collect the user set.

\subsection{Analysis performed on collected data}

For each user, a profile was generated using our profilers for both Twitter and
YouTube services. We measured effectiveness of profilers as a ratio of number of
items (tweets/videos) with linked data concepts identified to the overall number
of items analyzed. The results are presented in the table below.

\begin{center}
  \begin{tabular}{| l | l | l |}
  Topic & Twitter & YouTube \\ \hline
  Film actors & 0.48\% & 13.12\% \\
  Movie titles & 1.79\% & 45.95\% \\
  TV actors & 0.20\% & 14.73\% \\
  TV shows & 1.5\% & 30.25\% \\
  \end{tabular} \\
  Figure 5: The comparison of the amount of linked data concepts within separate data streams \\
\end{center}

The reader should note, there was no checking of actual correctness of those results at
this stage (the evaluation of results using surveys is presented in the next chapter). We
can still draw some conclusions after this step.

\subsection{Use of text in social services}
YouTube looks to be a more promising area for this kind of search. The
linked data concepts were present in up to 50 times as many items as for the other
web service for every type of concepts searched. This is most probably a
result of different natures of text data in both services.  

Twitter web service doesn't encourage its users to publish their updates in any
specific form. The single requirement is that the length of a single message
was less than 140 characters. For a video description on YouTube: even the name
suggests that this field should be used to characterize the contents
of a video. Such a difference in suggested use might explain the fact, that
much more named entities were found in YouTube stream. Also, a relatively
unrestrained form of Twitter updates is probably the reason for a lower rate of
concepts found in data coming from this web service. Often, instead of properly
mentioning an entity by name Twitter user would decide to use only person's
last name of an acronym of a TV show's title. Behaviours like that result in a
lower effectiveness of text-based entity finder.

\subsection{Variety across concept types}
Another remark worth making is that number of matched items depends on type of
concepts that we are in search of. However, after manual analysis of identified
items, authors came to a conclusion that comparing data stream pervasiveness of
different concepts is counterproductive at this stage. And that is because of
numbers of false positives, which vary wildly depending on the topic. This
observation will be extended in the next chapter.

\subsection{Dependency on amount and quality of data}

The efectiveness of profiler -- not surprisingly -- depends strictly on data
that the profiler was fed. Only a small number of users -- those, whose updates
are published frequently and often feature media-related subjects -- could be
profiled thoroughly. Profiles for similarily small group should be considered
to be only stubs, which would be of relatively little use for recommenders.
Profiles for the rest of users would be essentialy worthless. A technique that
might amortise the effect of small amounts of data would be combining results
coming from various web services. This way, in face of scarce YouTube data, the
profiler might still return meaningful results based on data coming from
Twitter.
