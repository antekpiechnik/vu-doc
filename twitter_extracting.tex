\section{Extracting user profile from Twitter}

\subsection{Approach}
In order for the results we decided to be dividing the corpus into two halves.
By measuring the difference in achieved results, we might be actually able to
evaluate their relative usefulness.

\paragraph{Full name matching}
Mentions a full name (for all actors and also TV shows/movies titles longer
than one word). The figure show's relative amount of tweets containing those
matches out of all tweets browsed.

\begin{center}
  \begin{tabular}{ | p{4cm} | p{2cm} | p{1cm}| p{2cm} | } \hline
    Entity (average) & Corpus 1 & & Corpus 2 \\ \hline
    TV Shows & 1.24\% & & 1.76\% \\ \hline
    TV Actors & 0.41\% & & 0.00\% \\ \hline
    Movies & 1.92\% & & 1.69\% \\ \hline
    Movie actors & 0.74\% & & 0.21\% \\ \hline
  \end{tabular}
\end{center}

This score should be considered a baseline for further research. Those low scores seem to indicate
that mentioning those entity names is a sign of interest.

\paragraph{Matching twitter usernames}
Due to poor availability of Twitter usernames for various entities (we were able to locate less than 200 Twitter
usernames related to TV media), this approach seems not to be effective. Including those usernames adding them to the pool,
has brought extremely little (0.02\% in on of the corpuses for TV Shows) to almost none increase in the results.

\paragraph{Matching title acronyms}
Only entities with a title with 3 or more words have been used for this measurement.

\begin{center}
  \begin{tabular}{ | p{4cm} | p{2cm} | p{1cm}| p{2cm} | } \hline
    Entity (average) & Corpus 1 & & Corpus 2 \\ \hline
    TV Shows & 3.07\% & & 2.17\% \\ \hline
    Movies & 2.01\% & & 2.33\% \\ \hline
  \end{tabular}
\end{center}

We can easily spot an increase in matched entities, however there might be many
misleading acronyms created with this approach. Acronyms such as \textit{SNL}
(for \textit{Saturday Night Live} show) or  \textit{BBT} (\textit{Big Bang
Theory}) are widely used, probably even as much as the full titles. However,
if a \textit{CAT} acronym emerges, it will drastically increase the noise generated
by the matching algorithms (thus the greatly increased percentage of tweets found).

\paragraph{Matching name converted to a hashtag form}
All entities have been converted to a hashtag form (such as
\textit{\#SaturdayNightLive} for Saturday Night Live). Also one-word entity
names have been used.

\begin{center}
  \begin{tabular}{ | p{4cm} | p{2cm} | p{1cm}| p{2cm} | } \hline
    Entity (average) & Corpus 1 & & Corpus 2 \\ \hline
    TV Shows & 1.18\% & & 2.21 \% \\ \hline
    TV Actors & 0.00\% & & 0.03\% \\ \hline
    Movies & 2.49\% & & 3.12\% \\ \hline
    Movie actors & 0.67\% & & 0.09\% \\ \hline
  \end{tabular}
\end{center}

We can easily notice how the person-based entities have either noted a smaller
occurrence rate and title-based ones have improved. This may be related to the
fact that people usually create hashtags based on a more general entity (such as
a movie) rather then it's specific parts (such as actors that play in it)
when expressing their opinion.

\paragraph{Usage of activity verbs}
A rather small activity verbs vocabulary has been used.
\\ Below is a figure showing percentage of tweets using an activity verb
and a entity name (full match) out of all that have been matched with an
entity's name (hence the greater percentage scores).

\begin{center}
  \begin{tabular}{ | p{4cm} | p{2cm} | p{1cm}| p{2cm} | } \hline
    Entity (average) & Corpus 1 & & Corpus 2 \\ \hline
    Movies & 37.4\% & & 24.3\% \\ \hline
    TV Shows & 21.2\% & & 13.4\% \\ \hline
    TV Actors & 1.3\% & & 0.6\% \\ \hline
    Movie actors & 2.1\% & & 0.0\% \\ \hline
  \end{tabular}
\end{center}

As we can see, the activity verbs are very unlikely to be occurring next to
person-based entities. However, activity verbs are much more popular with both
Movies and TV Shows, which originates from the very idea of Twitter as
updating statuses with information on what the user is currently doing rather
than expressing their opinions on various subjects.

\paragraph{Usage of preference verbs}
Two sets of preference have been used:
\begin{itemize}
\item positive
\item negative
\end{itemize}

The figure below shows the general use of preference verbs with occurrences of
mentions.

\begin{center}
  \begin{tabular}{ | p{4cm} | p{2cm} | p{1cm}| p{2cm} | } \hline
    Entity (average) & Corpus 1 & & Corpus 2 \\ \hline
    TV Shows & 7.4\% & & 6.3\% \\ \hline
    Movies & 9.1\% & & 8.4\% \\ \hline
    TV Actors & 0.0\% & & 0.9\% \\ \hline
    Movie actors & 1.2\% & & 2.7\% \\ \hline
  \end{tabular}
\end{center}

And a positive-to-negative preference ratio:

\begin{center}
  \begin{tabular}{ | p{3cm}| p{2cm} | } \hline
    Type & Amount \\ \hline
    Positive & 92\% \\ \hline
    Negative & 8\% \\ \hline
  \end{tabular}
\end{center}

The amount of preference verbs used whilst mentioning an entity is definitely
smaller compared to activity verbs. However, the Positive-to-Negative ratio most
certainly suggests that users' media preferences expressed on Twitter are
usually positive.

\paragraph{Relation between preferences and following}
By gathering available Twitter usernames from the \textit{Freebase} database,
we were able to perform searches for mentions of those usernames within the followers' streams.

The following figure shows the share of different kind of mentions of the specific entity while following.

\begin{center}
  \begin{tabular}{ | p{3cm}| p{2cm} | } \hline
    Match type & Occurence \\ \hline
    Name & 0\% \\ \hline
    Hashtag & 6\% \\ \hline
	Username & 94\% \\ \hline
  \end{tabular}
\end{center}

Since Twitter usernames mostly represent specific people rather than any other kinds of entities, it seems as if users mostly
mention them using their usernames rather than their names in plain or hashtag form.

However, those mentions occur relatively rarely. For a sample of 70 twitter usernames and 5 followers each
(around 4000 tweets), we were able to find out only 24 tweets (about 0.57\%) mentioning directly the people they are following.

Furthermore, following a certain entity should also be regarded just as a preference toward the topics this entity is connected to (such as
Politics for \textit{BarackObama}).

On the other hand, locating more \textit{official} twitter usernames for various entities is challenging. Making this approach not
completely worthwhile. Given such small amounts of twitter usernames known, it would be hard to reasonably specify the
relation between the act of following the entity on twitter and preferences towards it, thus making in the profiling
based on the ones known less influential than the other methods.

\subsubsection{Summary}
All the methods used are slightly naive and may not always reflect the actual
state of preferences. Not only do some names of entities not actually always
represent what we would like them to.
\\ What is more, due to a very specific nature of the
tweets and their structure, it is quite hard to apply more definite Natural Language Processing
methods. Sentence in tweets mostly do not contain a proper object making it quite hard to
relate the activity/preference verb to the entity itself.

\subsection{Counting methods applied in the profiling.}

\subsubsection{Counting matches to entity names (without the preference/activity verbs,
    suggests mostly the interest in the entity)}

Due to possible misleading suggestions, this counting method will have a low value of interest.
However, the more a specified subject occurs in a stream, the greater the preference in it will be.

\subsubsection{Counting matches to names in hashtag forms (seeming more significant than
      just entity names)}

Since a hashtag might suggest a specific topic being mentioned (such as a TV Show or a Movie), the chance
of the matcher to be wrong is much lower. It is still however a low-success rate matcher
and thus the profiling it generates will be of a smaller value.

\subsubsection{Counting entity names with activity verbs (suggesting that the topic has
    actually been seen)}

This part will provide profiling based on tweets including a verb suggesting
having actually seen the given entity. It will have a much higher preference
value than the name matching. Since an interested user might tweet repeatedly,
the profiling value will increase with each such mention. \\
This value might also be modified by the amount of different entities found,
(the more user tweets about the TV, the more its significance differs)

\subsubsection{Counting entity names with preference verbs (using a greater vocabulary
      and waging specific interests)}

The extraction of the mentions will be similar to the activity verbs. However,
relating a specific preference verb to the entity with a 100\% accuracy. The
profiling influence will be matched accordingly. \\
Since users tend to tweet about positive preferences, the occurrences of negative
preference verbs will have a different effect on the profile.

\subsubsection{Measuring the amount of tweets mentioning any kind of entity to all
  tweets}

Such measurement may inform us as to how often a user tweets about TV-related
topics, giving us a more general information on his interest in TV.

\subsubsection{Approach}
We have used all above methods excluding the last metric. For now, only Movie titles, actors,
TV actors and shows have been used.

Initial results show that even using simple matching methods we are able to extract interests from
user's Twitter stream. Given that the user is a frequent user and updater of his Twitter stream, it seems
as if the average amount of tweets from which any kind of such data might be extracted oscillates around \textit{3-18\%}.

However, there is a great number of entity names (mostly TV Shows) that create noise around the results (shows such as \textit{Me too}).

The results seem to be very promising and indicate that data regarding the media preferences of a user might
be available in their Twitter stream and is also quite easily accessible.

The analysis of the \textit{following} list could also help profiling more efficiently when it comes to topics that the
given user is interested in.

\subsection{Additional problems/ideas}
\paragraph{Accidental matches}
In order to reduce the chance accidental name matches occurring. Entity names should be ran against a corpus of English texts,
ranking them by their popularity. The more often an entity name occurs in those corpora, the lesser the chance of it occurring
as a name of the TV shows. In order to measure the certainty, a separate function should be introduced, taking into account the following:

\begin{itemize}
  \item form of occurrence of the entity name (string match, hashtag)
  \item use of an activity verb
  \item use of a preference verb
  \item the popularity of the entity name in natural texts
\end{itemize}

This measurement will then be easily translatable into the preference weights to be recorded when profiling a user.
In order to further minimize the chance of finding an occurrence of a subject that is not describing what we are hoping is analyzing the tweet's context.
By undertaking semantic approaches we could be able to analyze different topics found in a tweet and attempt to rank their relation to the entity we are researching.

\paragraph{Context}
In order to further minimize the chance of finding an occurrence of a subject that is not describing what we are hoping is analyzing the tweet's context.
By undertaking semantic approaches we could be able to analyze different topics found in a tweet and attempt to rank their relation to the entity we are researching.

\paragraph{Ambiguity}
Since some entity categories (such as \textit{TV Actors}) can be ambiguous (such as \textbf{John Terry the actor} and \textbf{John Terry the footballer}). Such situations
would require us to use additional methods (such as context-based reasoning) in order to accurately identify the entity in question.

