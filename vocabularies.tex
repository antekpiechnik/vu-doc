\section{Vocabulary}

In order to make the user profile compatible with linked data sources, we used
vocabularies of semantic data from the \textit{Freebase}\footnote[1]{http://freebase.com} service, due to it's
well-organized and easy accessible API.

Freebase is a service offering an enormous (over 10 milion topics) amount
of semantically linked data on various topics.

The data available on Freebase is generally divided into topics, types and properties. Topics correspond to data freely
available e.g. on \textit{Wikipedia}, describing physical entities, artistic creations, abstract concepts etc.

Types, on the other hand, refer to different aspects of the same topic (e.g. Bob Dylan is a song writer,
singer and a film actor which all represent different types). All types have properties referring those
types to the topics (such as songs that Bob Dylan has written).

All topics and types have IDs contained within different domains (such as /en/tina\_fey for actress Tina Fey
or /medicine/disease for the type Disease). By referring to those IDs a user is able to search Freebase for
more linked data regarding the interesting topic and type.

From the whole wealth of data, we have focused only on the data related to the media, such as TV Actors, Movies, Feature
actors and TV Programmes as well as Bands and Singers' names.