\section{Vocabulary}
\label{sec:vocabulary}
In this section we would like to describe how vocabularies are used to help recognize \textit{named entities} (NEs)
known people shows and programmes within the available UAD for \textit{YouTube} and \textit{Twitter}.

\paragraph{Dataset used}
In order to link the concepts in the user profile to known and uniquely identifiable concepts in the Linked Open Data
cloud\footnote{http://linkeddata.org}, we used vocabularies from the \textit{Freebase} service.

Freebase is a service offering a large (over 10 million topics) amount of semantically linked data on various topics.

The data available on Freebase is divided into \textit{topics}, types and properties. \textit{Topics} correspond to data
freely available, similar to \textit{Wikipedia}, describing physical entities, artistic creations, abstract concepts
etc. In comparison to \textit{Wikipedia}, all of the data on \textit{Freebase} is semantically structured. Moreover,
users are allowed to edit, add and remove data. As mentioned, the \textit{Freebase} project is a part of Linked Open Data.

\textit{Types} refer to different aspects of the same topic, \eg \textit{Bob Dylan} has several types, namely
song writer, singer and film actor. Types have properties relating topics of a certain type to other topics, \eg
\textit{Bob Dylan} wrote some songs and played in a movie.

All topics and types have IDs expandable into URIs pointing to topic location in the \textit{Freebase} dataset,
such as \textit{/en/tina\_fey}\footnote{www.freebase.com/view/en/tina\_fey} for actress \textit{Tina Fey}
or \textit{/medicine/disease}\footnote{www.freebase.com/view/medicine/disease} for the type \textit{Disease}.
By referring to those IDs a user is able to search Freebase for more linked data regarding the interesting topic and type.

From the whole wealth of data, we have focused only on the data related to the media, namely TV Actors, Movies, Movie
actors, TV Programmes and Bands and Singers' names.

\paragraph{Usage in the research}
We have decided to use the \textit{Freebase} dataset in order to be able to uniquely identify mentions of different NEs
within the available UAD. This approach is beneficial for use with recommender systems (such as the recommender
in the \textit{NoTube} project), enabling them to immediately use the wealth of relational semantic data available
in the \textit{Linked Open Data cloud} for expanding their recommendations to related concepts.
