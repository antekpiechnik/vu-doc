\section{Extracting user profile from YouTube}

Our goal is to extract a user profile. For this paper, we will assume that user
profile is a collection of linked data concepts, that should be understood as
user's interests. Many YouTube videos have very semantic meaning -- they can
be associated with existing concepts (like actors or music bands) by performing
text search. In addition to that, analysis of tags can give an overview of user's
interests. This kind of data is useful in two
ways: first, it lets easily identify interests for a user, second: it can be
used to verify interests generated with other methods..

In the following sections, we will describe methods used to generate the user
profile. First, we describe our tag counting approach and compare the possible
use of tags and categories in user profiling. Then we will discuss possible
additional information coming from repeating occurrences of the tags in
different sets of videos. The next subsection present the double
nestedness problem, which makes counting tags from subscriptions fundamentally
different that the other two sets. The last subsection describes matching
YouTube videos to actual linked data concepts.

\subsection{Counting YouTube tags and categories}
Analyzing tags information carries some problems. Tags of a single video clip
can be either too numerous, making many
of them irrelevant, too sparse, nonexistent and occasionally just wrong. To avoid
that problem, we aggregate the tags across sets of video clips related to the
profiled user (favourites, subscriptions, uploads). Similar aggregation was
performed for the videos' categories. Below is a comparison of results of counting
tags (chosen by the user) and categories (picked from a small predefined set).


\begin{table}[ht]
\begin{minipage}[b]{0.5\linewidth}\centering

\begin{tabular}{| l | l |}
Category & \# \\ \hline
News & 8 \\
Music & 5 \\
Education & 3 \\
Comedy & 2 \\
\end{tabular}

\end{minipage}
\hspace{0.5cm}
\begin{minipage}[b]{0.5\linewidth}
\centering

\begin{tabular}{| l | l |}
Tag & \# \\ \hline
sonik & 5 \\
bogusław & 5 \\
trzech & 5 \\
kupli & 5 \\
wybory & 4 \\
kraków & 4 \\
polska & 4 \\
\end{tabular}

\end{minipage}

\caption{Most popular tags and categories from account of Bogusław Sonik -- a
polish politician. The word ''wybory'' stands for ''election'' in polish, and
\emph{Trzech kumpli} is a title of a controversial political documentary.}
\end{table}


\begin{table}
\begin{minipage}[b]{0.5\linewidth}\centering

\begin{tabular}{| l | l |}
Category & \# \\ \hline
Music & 53 \\
Comedy & 20 \\
Entertainment & 8 \\
Games & 4 \\
People & 3 \\
Travel & 3 \\
News & 2 \\
\end{tabular}

\end{minipage}
\hspace{0.5cm}
\begin{minipage}[b]{0.5\linewidth}
\centering

\begin{tabular}{| l | l |}
Tag & \# \\ \hline
metal & 17 \\
music & 11 \\
rock & 11 \\
video & 7 \\
the & 7 \\
black & 7 \\
of & 7 \\
\end{tabular}

\end{minipage}

\caption{Most popular tags and categories from anonymous user -- apparently a
heavy metal music fan}
\end{table}

We can see that category count, even though providing general overview of
person's interests (and the ways she uses youtube), cannot be used to find
specific interests. We can say that a person is more keen to
favourite music videos or comedy videos after looking at his categories, but in
order to say what kind of music the person prefers, we need more specific
information gathered from (e.g.) tags.

\subsection{How to handle duplicates}

When combining tags collected from various sets of videos, a question should be
asked what preference -- if any -- should be given to tags appearing in more
then one of them. Let's analyze some examples. The letters $f$, $s$ and $u$ mean number
of videos in accordingly: favourites, subscriptions and uploads. The
combinations of letters indicate the size of the intersections of those sets.

\begin{tabular}{| l | l | l | l | l | l | l | l |}
user & $f$ & $s$ & $u$ & $fs$ & $fu$ & $su$ & $fsu$ \\ \hline
politician & 219 & 32321 & 357 & 116 & 59 & 156 & 34 \\
metal music fan & 1103 & 7570 & 73 & 533 & 27 & 44 & 26 \\
one of authors & 687 & 5553 & 10 & 199 & 2 & 2 & 1 \\
\end{tabular}

The tags that appeared in all three sets are:
\begin{itemize}
  \item{politician: \emph{wybory, europarlament, czerwca, europa, europejska,
  parlament, 18, unia, europejski, kraków, w, 7, pe, 2009, bogusław, po,
  bezpieczeństwo}}
  \item{metal music fan: \emph{head, death, pantera, for, in, nosturi, the, live, cover, metal}}
  \item{one of authors: \emph{film}}
\end{itemize}

As expected, the tags in the intersection of sets give much more precise view of person's
interests as compared to tags occurring in a single set. Tags get muted when a user is not active in one of
those areas (e.g. does not upload movies very frequently). In order to minimize it's effect on the
end result, weights for different sets may be used, and we can maximize weights for tags occurring
in multiple sets.

\subsection{The double nestedness problem}
Gathering tags across user's favourites or uploads let us treat the whole video
set as a single document, giving each tag equal weight in the counting process.
Taking a similar approach for subscriptions (e.g. giving equal weight to each
tag of each video of every subscription) leads to some unwanted effects.

Suppose a user subscribes to two channels. One represents the british royal
family (133 uploads), the other one: Apple Inc., an american computer company
(19 uploads). The tags referring to new hardware releases would get flooded with
tags regarding british-specific content. In effect, the user's profile would
depend on third parties' youtube activity which is doubtful to be of any
importance to her.

In order to counteract this effect a number of techniques can be used. One of
them is giving tags in each subscription a weight, based on the number of videos or
total number of tags.

\subsection{Identifying entities in youtube streams}
The linked data concepts can be associated with user data through the use of
simple text search. The video's title and description fields often carry names
of people depicted in the clip. Having a set of all actors' names we can
identify their occurrences. Below are sample results for text-search-based
identification of actors among user's favourites.

\begin{center}
  \begin{tabular}{ | p{7cm} | p{4cm} | } \hline
    Title of video & Actor name in video data \\ \hline
    Rowan Atkinson LIVE: 02 - Fatal Beating & Rowan Atkinson \\ \hline
    Yes We Can - Barack Obama Music Video & Scarlett Johansson \\ \hline
    Compay Segundo - Chan Chan & Compay Segundo \\ \hline
    Drivin' Me Wild - Common Ft Lily Allen [OFFICIAL] & Lily Allen \\ \hline
    PT 4 - BRITNEY SPEARS 2006 INTERVIEW BEARS ALL & Britney Spears \\ \hline
    16-Lovestoned Live Futuresex/Loveshow & Justin Timberlake \\ \hline
    15- Cry Me A River Live Futuresex/Loveshow & Justin Timberlake \\ \hline
    Jennifer Lopez - Ain't It Funny & Jennifer Lopez \\ \hline
    Christina Aguilera - Save Me From Myself [Official & Christina Aguilera \\ \hline
    Nancy Sinatra Bang Bang & Nancy Sinatra \\ \hline
    Eddie Vedder - Hard Sun (Official Video) & Eddie Vedder \\ \hline
  \end{tabular}
\end{center}

This approach is not ideal. Many false positives can be returned, particularily
for concepts that are also common phrases (e.g. the "Lost" TV show). However,
above certain threshold of repeats, an
assumption that a user likes a given actor becomes more probable.

\subsection{Pervasiveness of linked data concepts}
A set of statistics was prepare to measure the pervasiveness of linked data
concepts as extracted from the freebase vocabularies. The matching performed
was a simple text-based search in videos titles and descriptions.

\begin{center}
  \begin{table}
    \begin{tabular}{ | p{4cm} | p{6cm} | } \hline
      Entity (average) & Rate of videos with matched contents 1 \\ \hline
      TV Shows & 30.25\% \\ \hline
      TV Actors & 14.73\% \\ \hline
      Movies & 45.95\% \\ \hline
      Movie actors & 13.12\% \\ \hline
    \end{tabular}
    \caption{Rate of video descriptions with entities found using text-based matching.}
  \end{table}
\end{center}

These numbers are exceptionally large, but unfortunately, a vast number of text
based matches proved to be false positives. This was the case particularily for
movie titles, where entities with common names are quite popular (e.g. 'Under',
'House', or 'Power').
